% !TEX root = ../calibreport.tex
%===========================================
%===========================================
\section{Objective}
%===========================================


This study is devoted to the calibration of an infrasound system, which consists of its sensor and its Wind Noise Reduction System and which is assumed to be a linear time-invariant system, commonly called a linear filter in the literature.  It is expected that most results could be applied to the calibration of seismometers, using a reference instrument.\\*

First, it is interesting to note that a failure on the observation system can not be an isolated event of short duration. When a sensor failure occurs, it occurs for a long period of time. It follows that a malfunction of short duration can be regarded as a particular case and can be removed from the useful data.\\*



The specifications set out in the Operational Manual \footnote{CTBT/WGB/TL-11,17/17/REV.5 Operational Manual for Infrasound Monitoring and the International Exchange of Infrasound Data} provide Minimum Requirements for the calibration of Infrasound Stations: the frequency response shall be within $\pm 5\%$ in amplitude gain from the nominal response in the PTS passband ($0.02$ Hz to $4$ Hz). That represents a challenging issue. 
It should be noted that no Minimum Requirements are listed for the phase response of Infrasound Stations, yet it is stated in part 4.5 Calibration of the Operational Manual, that: "The calibration minimum requirements for infrasound stations do not currently include phase measurement; however, these measurements are necessary to establish the full system response that is required for data processing at the International Data Centre."\\*


%================ figure ====================
\figscale{schema-of-observation-simple.pdf}{Measurement process. SUT denotes the sensor under test, it consists of a sensor and its noise reduction system. SREF denotes the reference sensor. SUT and SREF are assumed to be linear filters. $e_{u}(t)$ and $e_{u}(t)$ denote the non observable input signals. $x_{u}(t)$ and $x_{r}(t)$ are sampled and recorded.}{fig:schema-of-observation-simple}{0.8}


The method to estimate the response of a infrasound system, denoted System Under Test (SUT) in the following, is based on the use of a second sensor, denoted System of Reference (SREF), whose frequency response is perfectly known. The calibration method is based on the presence of a quasi permanent background signal all over the Word \cite{?}.  Still in \cite{?}, experimental results have shown that this background signal has a large enough frequency bandwidth and energy to satisfy the PTS requirements. \\*


This  background signal  induces two  signals denoted $e_{u}(t)$ and $e_{r}(t)$ in the figure \ref{fig:schema-of-observation-simple}
A useful property is that a large part of these two signals is simultaneously present  on the two sensors. The main reason of the presence of this same signal is related to the spatial coherence which is due to the fact that the signal wavelength is much larger than the distances  involved in the measurement process: for a common acoustical velocity of $300$ m/s, and the highest frequency of interest saying $4$ Hz, the wavelength is $75$ m which is much larger than the distance between the two sensors, which is less than 3 meters. 
\change[momo]{}{It is worth to notice that the air turbulences, which can be at first order characterized by the ratio $v/f$ where $v$ is its velocity of around a few meters per second, e.g. $4$ m/s, induces spatial coherence phenomena for frequencies under $f<0.02$ Hz, see chapter ...}

\change[momo]{}{In the following the common part, denoted $s(t)$, is shortly called the \emph{coherent signal}. However spatial coherence does not mean that there is a commun part for the inputs of the two sensors. It means that a part of the SUT signal is a filtering of a part of the SREF signal. But if such situation occurs, the filter involved is not identifiable. In the following we assume that this filter is the identity filter.  More specifically we let:}
\begin{eqnarray}
\label{eq:model-signal-introduction}
\left\{
\renewcommand\arraystretch{1.6}
\begin{array}{rcl}
e_{\ut}(t)&=&s(t)+w_{\ut}(t)
\\
e_{\rf}(t)&=&s(t)+w_{\rf}(t)
\end{array}
\right.
\end{eqnarray}
where $w_{\ut}(t)$ and $w_{\rf}(t)$ are such that the correlation between them are zero for any couple of times. In the following they are shortly called \emph{noise}.  They are mainly due to the wind.\\*

Unfortunately the signal $s(t)$ is not  \emph{observable}. It follows that, in presence of the additive noises $w_{\ut}(t)$ and $w_{\rf}(t)$, the problem is ill-conditioned in the sense that they are an infinity of possible solutions.






%Although the wind structure is very different of the  "acoustical'' signal $s(t)$, we will see that a parameter of interest is $\zeta=v/f$ where $v$ is the wind velocity and $f$ the frequency. $\zeta$ can be interpreted as a wavelength. \annote[Benoit]{For typical values of the wind velocity, $\zeta$ is much lower than the distances between air inlets of the noise reduction system (NRS).}{Je ne comprends pas bien l'int\'er\^et de cette phrase, surtout qu'a mon sens, en basse frequence le ratio $\zeta$ est presque toujours plus grand que la distance inter-inlet. Un aspect du probl\`eme consiste justement \`a comprendre et mod\'eliser le "dip" comme une fonction de $\zeta$, non ?}

%===========================================
 \newpage\clearpage
%===========================================
\section{Experimental testbed}
%===========================================
In may 2015, an experimental setup has been deployed in the IS26 located in Freyung in Germany:
\begin{itemize}
\item
The different sensors under test/reference are reported table \ref{tab:sensor-specifications}. Each reference sensor was installed in the same vault as the sensor under test, and was connected to a newly installed reference inlet through a short pipe. The reference inlet port was positioned at a short distance (less than a few meters) from the main manifold of the Wind Noise Reduction System. 
These sensors have been checked before installation. We have two kinds of SREF, one is MB2005, the other MB3. The SUT sensor is a MB3. For our calibration problem both SREF sensors can be considered as almost identical.
\item
The environment is particularly  quiet, due to the Bavarian Forest where the station is located. It follows that the wind is very low during large periods of time.

 \item
The data are recorded in continuous at the IDC, in testbed.
\end{itemize}

It is worth to notice that the calibration problem is a little bit different of the detection problem which consists to provide an alert when the system is out of specifications.


\begin{table}[h]
\begin{center}
\begin{tabular}{|c||cc|cc||}
\hline
Site & SUT & info.& SREF & info.
\\
\hline
     & model&serie \#     & model&serie \#
\\
H1/C1& MB3 & 00020  & MB2005& 6046
\\
H2/C2& MB3  &  00017  & MB2005& 5125
\\
H3/C3& MB3  &  00014  & MB2005 &6039
\\
H4/C4& MB3  &  00023  & MB2005& 5104
\\
H5/C5& MB3   & 00012  & MB2005 &7095
\\
H6/C6& MB3  &  00011  & MB3 &00007
\\
H7/C7& MB3   & 00021  & MB3 &00008
\\
H8/C8& MB3   & 00015  & MB3 &00022
\\
\hline
\end{tabular}
\parbox{12 cm}
{
    \caption{\protect\small\it  sensor specifications: all sensors have the same theoretical sensitivity of $0.02$ V/Pa. The MB3s have self noise lower than that of the MB2005s. The sites associated to the ''Wind Noise Reduction System" are labelled H whereas  the sites of reference sensors are labelled C. In the site 1, we have wind informations, as direction, velocity, etc}
    \label{tab:sensor-specifications}
}
\end{center}
\end{table}

%\figscale{H1-H5MB2005.jpg}{}{}{fig:H1-H5MB2005}

 \newpage
%===========================================
%===========================================
\section{Main issue}
%===========================================

The main issue is related to the under-determination of the problem (also known as blind identification). More specifically in our model we have four unknowns for only two observations. The four unknowns are the frequency response of the SUT, the spectral shape of the coherent signal and the spectral shapes of the two noises. The two observations are the output signals on both sensors, see figure \ref{fig:general-schema}.

A common way to solve this kind of problem is to introduce some {\it a priori} knowledges/assumptions. The drawback is what happens when the {\it a priori} knowledges are not well-verified.

Here a list of {\it a priori} knowledges that could be considered:
\begin{itemize}
\item
 the signals are stationary in the whole frequency bandwidth of interest. More specifically we have to precise what we mean in terms of stationarity duration. The worst case  is for the low frequency, i.e. for long period. For example if the target accuracy requires to integrate the signal over five periods and if we want to analyze the spectral content around the frequency of $0.01$ Hz, about $10$ minutes of stationarity are needed.
 
Let us notice that, even under the stationarity assumption the problem is still under-determined.
 

  \item
the coherent signal and the noises can be modeled with parametrical models such as auto-regressive paradigm. In \cite{frazier:2013} a generalized autoregressive conditionally heteroskedastic (GARCH) model has been used to model the wind. This approach has not been investigated here.

  \item
 the noises on the two sensors are uncorrelated and white. This is the simplest parametrical model which depends on only one parameter. In this case the frequency response is identifiable, but observations and also the physical aspects of the wind noises deny this assumption.
   \item
the system under test is a linear filter with {\it given} numbers of poles and zeroes (parametrical approach). This approach has not been already investigated. It follows that, for the calibration, implying the numbers of poles and zeroes could hide a singularity and then induce a bias. To illustrate this statement, saying that a system is of order 1 can hide in the estimated response the presence of a non suspected resonance. \annote[Benoit]{But in the alert issue framework, we  see that something is wrong, therefore we suspect that it could be an efficient way for test.}{A reformuler / preciser.} 

\item
assuming that the two noises are uncorrelated, the indetermination disappears if  we assume that the ratio between the two noise levels is known. That is a realistic way in high frequency because the NRS works well and the ratio is about the inverse of the number of inlets.

\item
assuming that the two noises are uncorrelated, the indetermination disappears if  we assume that one of the two noises is negligible, typically it could be the case for the SUT which is provided with a noise reduction system (NRS). But we also want to detect if this NRS is working well.

The advantage of the  assumption is that the second noise could be very large, we only have to use the formula \eqref{eq:ratio-sup}. The drawback is that no test does exist to provide an answer to the question: is the noise on the SUT negligible? Therefore if the assumption becomes wrong the error could be very large.

\item
assuming that the two noises are uncorrelated, the indetermination can be a fortiori removed if both noises are negligible. The advantage of this second assumption is double: (i) we can use any of the two formulas \eqref{eq:ratio-sup} or \eqref{eq:ratio-inf}, and (ii) we have an efficient way to test if or not the two noises are negligible. The main drawback is it could be difficult to find time windows where the condition is verified for a longperiod, particularly in the high frequency band. A solution is to consider shorter window sizes in the high frequency ranges. 



\end{itemize}

In summary, our conclusion is that, to get the expected accuracy, we have to consider that both noises are negligible. To test that the MSC (see below for specific definition) is used along time windows whose lengths vary in relation with the frequency ranges. From picturing manner, the MSC measures the fact that the estimated values exhibit very low dispersion, hence low noises, along a few consecutive windows. 
 
