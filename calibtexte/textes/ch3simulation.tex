% !TEX root = ../calibreport.tex

%======================================
To investigate the effect the approach, we focuse at first on a short simulation in such a way the groundtruth is available. For that, following the notations of \eqref{eq:model-of-obervation} we take, for $s(t)$, about half an hour of a MB3 sensor record. The signal has a spectrum reported on figure \ref{fig:zeronoideonsutwhiteSOI}. We see that the signal contains the most power in the low frequencies.

There is no filtering on the SREF signal, more specifically $G_{\rf,k}=1$.  The filter on the SUT signal is a FIR filter whose transfer function consists of a numerator with 3 coefficients and a denominator with 3 coefficients, whose values are reported on table \ref{tab:UT-sensor}
\begin{table}[h]
\begin{center}
\begin{tabular}{|c| c c c||}
\hline
numerator  & $1$ &$-1.0$ &$0.35$
\\
denominator  &$1$ &$-1.15$& $0.45$  
\\
\hline
\end{tabular}
\parbox{12 cm}
{
    \caption{\protect\small\it  Coefficient of the IIR(2,2) used in the simulation for the SUT. The SREF  has a gain 1. It follows a ratio which presents a large dynamic. }
    \label {tab:UT-sensor}
}
\end{center}
\end{table}


The wind noise is modeled as the output of the a FIR filter whose input is a white sequence. The FIR filter impulse response consists of $4$ one's. That corresponds to a low-pass-filter in such a way the synthetic noise does not introduce much poorer SNR on the high frequency band, regarding the spectrum of $s(t)$.

The spectra are performed using Welch's approach. The basic scheme considers the averaging of successive discrete Fourier transform (DFT) with 50\% of overlapping. The maximal size of the DFT is related to the long period signals. Typically in our case that is of 50 seconds. The number of successive windows is related on the mean square error of the spectral estimation. Greater the number of windows better the accuracy, but to greater can lead to the loss of stationarity. We opt to a number of windows corresponding at 10 times the DFT size.

We definitively opt for a Hann's window which provides a good compromise between resolution and leakage. More specifically the leakage (which is the effect of transferring energy from high energy frequency band to low energy frequency band) is low enough to exhibit the presence of microbarom around 0.2 Hz. 

%===========================================
\section{Zero-noise on the SUT}
%===========================================
On the simulation we see at first that the efficiency of the estimator given by the expression \eqref{eq:ratio-sup} does not present a visible bias. Also when we reduce the time period of estimation we improve the results in the high frequency range.

The drawback is when we introduce noise on the SUT, the formula is no more valid (underdetermination) and much more we can not test the presence/absence of the noise. Moreover if there is no noise on the SUT, hence the formula does not present a bias but does present a variance, which can be outside the PTS specifications.

As a conclusion this approach is not advised for the calibration. 


%===========================================
\section{Zero-noise on both SUT and SREF}
%===========================================
There is negligible noises on both sensors. All results are reported on figures \ref{fig:zeronoideonsutwhiteSOI} and \ref{fig:zeronoideonsutrealSOI}.

We have chosen two estimation periods, 250 seconds and 50 seconds for white coherent signal and real coherent  signal. For each the time are slotted in 9 windows with 50\% of overlap to performe FFTs.

As expected we see figure \ref{fig:zeronoideonsutwhiteSOI} that the two periods provide similar results, regarding the coherent signal is stationary and white all over the frequency bandwidth.

Figure \ref{fig:zeronoideonsutrealSOI} we see some differences on the two periods. Indeed the coherent signal, extracted from the MB3 (H8/C8), may be viewed as consisting of a certain loss of stationarity. So it appears that, in the high frequencies,  on the shorter period case the coherence increases, hence we can find several time slots with MSC over the threshold, not detected with longer time slots. However it is worth to notice that some high values of the MSC come from the variability of the estimator and not on the loss of stationarity.

In summary it is possible, if necessary, to reduce the period in high frequencies to detect more time slots if the stationarity duration is as the inverse of the frequency location. Greater the frequency, shorter the stationarity duration. We have not validate this assumption. 

To use this approach we need an MSC above 0.99. What we see is very typical. If we use long period, to be able to estimate the ratio at low frequency, many values of the estimated MSC are under this threshold. But if we use two different length, we can access in high frequency domain to a greater number where the MSC is still over this threshold. 

In conclusion we propose to combine long period/short period to estimate the ratio in the frequency range of interest. It is worth to notice that the short period is used to improve the MSC estimation but to avoid a loss of stationarity.

 \figscale{zeronoiseinbothsensorswithwhiteSOI30dB.pdf}{White ({\tt randn} from Matlab) signal with SNR of $30$ dB on each sensor. Black curve, in the mid-curves is the true ratio. Red curve in the bottom is the true MSC. The true MSC curves present a large variability, because we use a real signal as signal of interest and we need, at first, to estimate its spectrum $\gamma_{ss}$ and then use the formula \eqref{eq:coherence-in-our-model}.}{fig:zeronoideonsutwhiteSOI}{0.7}


%===========================================
\section{Comparison without/with filter bank}
%===========================================

Also we have reported figures \ref{fig:realSOIandFB} and \ref{fig:realSOInoFB} 10 simulations with/without filterb bank (see for filter bank table \ref{tab:freq-duration-tradeoff}). The selected MSC threshold is 0.99. It appears that the accuracy seems to be better in the case with filter bank.

  \figscale{zeronoiseinbothsensorswithrealSOIandFB.pdf}{Real SOI from MB3 with a six-filter bank. MSC threshold is $0.99$. Solid lines are the ground truth. Points correspond to 10 simulations.}{fig:realSOIandFB}{0.7}

  \figscale{zeronoiseinbothsensorswithrealSOInoFB.pdf}{Real SOI from MB3 with no filter bank. MSC threshold is $0.99$. Solid lines are the ground truth. Points correspond to 10 simulations.}{fig:realSOInoFB}{0.7}


 \figscale{zeronoiseinbothsensorswithwhiteSOIandFB.pdf}{White SOI (from {\tt randn}) with a six-filter bank. MSC threshold is $0.99$. Solid lines are the ground truth. Points correspond to 10 simulations.}{fig:whiteSOIandFB}{0.7}

  \figscale{zeronoiseinbothsensorswithwhiteSOInoFB.pdf}{White SOI (from {\tt randn}) with no filter bank. MSC threshold is $0.99$. Solid lines are the ground truth. Points correspond to 10 simulations.}{fig:whiteSOInoFB}{0.7}

