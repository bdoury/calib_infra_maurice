\documentclass[a4paper, 12pt]{report}
%================================================================
\usepackage{verbatim, fancyhdr, theorem, dsfont, color,
            amsmath, amsfonts, amssymb,
            hyperref,epsfig, graphicx, setspace}
\usepackage{graphics}
\usepackage[margins]{trackchanges}
% to insert pdf sous la forme
% \includepdf[pages=-]{../CoherePart_CR1nov2012/rapportCoherence.pdf}
%\usepackage{pdfpages}
%============================================
 \textheight 21cm
% \doublespace
%  \oddsidemargin -0.5cm
% \evensidemargin +1.5cm
%  \textwidth 17cm
% \topmargin 0cm
%============================================
% Figures
%============================================
\newcommand{\figsstit}[2]{
\begin{figure}[hbtp]
\centerline{
    \hbox{ \epsfig{figure={#1}, scale=#2} }
}
\end{figure}}
%============================================
\newcommand{\figscale}[4]{
\begin{figure}[hbtp]
\centerline{
    \hbox{ \includegraphics[scale=#4]{#1} }
}
\begin{center}
\parbox{12	 cm}
{
    \caption{\protect\small\it  {#2}}
    \label {#3}
}
\end{center}

\end{figure}}

%%============================================
%\newcommand{\figsstit}[2]{
%\begin{figure}[hbtp]
%\centerline{
%    \hbox{ \includegraphics{figure={#1}, scale=#2} }
%}
%\end{figure}}
%

%==============================================
\newcommand{\prob}[1]{\mathds{P}\left( #1 \right)}
\newcommand{\esp}[1]{\mathds{E}\left[ #1 \right]}
\newcommand{\var}[1]{\mathrm{var}\left( #1 \right)}
\newcommand{\cov}[1]{\mathrm{cov}\left( #1 \right)}
\newcommand{\diag}[1]{\mathrm{diag}\left( #1 \right)}
\newcommand{\trace}[1]{\mathrm{trace}\left( #1 \right)}
\newcommand{\card}[1]{\left| #1 \right|}
\newcommand{\myemph}[1]{\emph{\color{red}#1}}


 \newcommand{\benoit}[1]{\marginpar{\color{red}\footnotesize{BENOIT:}
                \color{red}\footnotesize{ #1}}}

 \newcommand{\momo}[1]{\marginpar{\color{blue}\footnotesize{MOMO:}
                \color{blue}\footnotesize{ #1}}}
%%============================================================================
%\def\thesection{\arabic{section}.}
%\def\thesubsection{\arabic{section}.\arabic{subsection}}
%\def\thesubsubsection{\arabic{section}.\arabic{subsection}.\arabic{subsubsection}}
%\def\thefigure{\arabic{figure}}
%\def\theequation{\arabic{equation}}
%\def\theexercice{\arabic{exercice}}
%\def\theexample{\arabic{example}}
%\def\theproof{\arabic{proof}}

%===============================================
\newtheorem{property}{Properties}
\newtheorem{remark}{Remark}
\newtheorem{theorem}{Theorem - \thetheoreme}
\newtheorem{definition}{Definition - \thedefinision}
\newtheorem{example}{Example}
\newtheorem{lemme}{Lemme - \thelemme}
\newtheorem{proof}{Proof - \theproof}
\newenvironment{TAB}{\begin{table}[[hbt] \center \leavevmode}{\end{table}}
%%=========== LOGO/Head/Foot par PAGE ========
%\pagestyle{fancy}
% \lhead[]{}
% \chead[]{}
% \rhead[]{\includegraphics[scale=0.1]{\DIRLOGO/logo_telecomparis}}
% \lfoot[]{\thechapter}
% \cfoot[]{}
% \rfoot[]{\thepage}

%%============================================================================
%\newcounter{auxiliaire}
%%%%%%%% comment
%\setcounter{auxiliaire}{\theenumi}
%\end{enumerate}
% TEXTE
%\begin{enumerate}
%\setcounter{enumi}{\theauxiliaire}
%%============================================================================
%\renewcommand\arraystretch{1.6}

\def\ua{\underline a}
\def\ub{\underline b}
\def\uB{\underline B}
\def\uH{\underline H}
\def\ur{\underline r}
\def\us{\underline s}
\def\ux{\underline x}
\def\uX{\underline X}
\def\uZ{\underline Z}
\def\utheta{\underline \theta}
\def\hat{\widehat}

\def\MSC{\mathrm{MSC}}
\def\SNR{\mathrm{SNR}}
\def\iSNR{\mathrm{SNR}^{-1}}

\def\degree{$^{\circ}$}

\def\ut{{u}}
\def\rf{{r}}

%============== colors ========================
\definecolor{enstrouge}{RGB}{212,65,84}
\definecolor{lightorange}{RGB}{235,226,52}
\definecolor{greennoise}{RGB}{243,42,255}
\definecolor{lightred}{RGB}{255,181,183}
\definecolor{light-grey}{rgb}{0.95,0.95,0.95}
\definecolor{peach}{rgb}{0.98,0.49,0.25}
\definecolor{burntorange}{rgb}{0.79,0.37,0}
\definecolor{light-yellow}{rgb}{1,1,0.92}

\definecolor{light-green}{RGB}{231,255,145}
\definecolor{enstorange}{RGB}{255,214,10}
\definecolor{enstrouge}{RGB}{212,65,84}
\definecolor{grey}{RGB}{204,204,204}
\definecolor{blue}{RGB}{0,0,255}
\definecolor{almost-black}{RGB}{100,100,100}
\definecolor{violet}{rgb}{0.4,0,0.4}
\definecolor{cyan}{RGB}{0,255,255}
\definecolor{magenta}{RGB}{243,42,255}

\def\degree{^{\circ}}
\def\simiid{\stackrel{\mathrm{i.i.d.}}{\sim}}
\def\MSC{\mathrm{MSC}}%{\MSC}}
\def\hMSC{\widehat{\MSC}}%{\MSC}}


 \def\programsfullprocess{fullprocess/}
 \def\programsToolbox{fullprocess/ZZtoolbox/}
 \def\programspierrick{fullprocess/ZZtoolbox/00Pierrick/}



\graphicspath{{figures/}}
%============================================
\begin{document}
 \sloppy
%  \nocite{*}
 \bibliographystyle{plain}
\tableofcontents
%===========================================
%===========================================
%============ part 1 ==========================
%===========================================
\part{Theoretical aspects}
%===========================================
\chapter{Problem position}
\input{textes/ch1introduction.tex}
%===========================================
\chapter{Spectral analysis for SUT estimation}
\label{chap:spectralanalysis}
% !TEX root = ../calibreport.tex
%===========================================


In this chapter we derive the expression of the three spectral components associated with the two outputs signals, as a function of the responses of the two sensors. Without \emph{a priori} knowledges  the problem is ill-conditioned. An approach based on the MSC can be used to remove the uncertainty. That leads to an estimation algorithm by replacing the spectral components by consistent estimates. These estimates are obtained via Welch's technic. Finally a filter bank approach is shown to be useful. Indeed spectral estimation  needs that the signals are stationary for long periods of time, which can be  very challenging particularly in the low frequency domain. For circumvent this difficulty we propose to separate the signals in different frequency bands.

\figscale{slidesITW2015/fulltestbedIS26.pdf}{The calibration systems consists of two channels: (i) the SUT with the noise reduction system (NRS), the sensor and the digitizer and (ii) the SREF with the sensor and the digitizer. Because the two digitizers are identical, the response ratio between the two channels does not depend on the digitizer. It follows that the response ratio corrected by the SREF response is an estimate of the SUT response, including the sensor and the NRS.}{}{1}


Therefore the main quantity of interest is the frequency response ratio between the two channels. It can be estimated by cross-correlation between the two outputs, assuming some hypotheses as stationarity.

\figscale{slidesITW2015/processdetail.pdf}{$\widehat{S_{UU}}(f)$, $\widehat{S_{RR}}(f)$, $\widehat{S_{UR}}(f)$ denote respectively the spectrum on channel SUT, the spectrum on channel SREF and the cross-spectrum between them. In the absence of noise the ratio between the 2 responses is given by the ratio $\widehat{S_{UU}}(f)/\widehat{S_{RR}}(f)$}{}{0.7}

%Let us remind the DFT expression, associated to the frequency $kF_s/N$, performing on $N$ samples denoted $x(0),\ldots,x(N-1)$:
%\begin{eqnarray}
%X_k&=& \frac{1}{\sqrt{N}}\sum_{k=0}^{N-1}x_n \, e^{-2j\pi kn/N}
%\end{eqnarray}
%where $k$ goes from $0$ to $N-1$.

 \newpage
\figscale{figures/synoptic.pdf}{The DFT buffer contains the data for a duration in accordance with the DFT analysis. Usually this duration is in the inverse of the bandwidth of the filter. The filter bandwidth is log-scaled in the frequency domain. If the SCP time window is $1000$ seconds and the DFT buffer duration is  a fifth i.e. $200$ seconds with an overlap of $50\%$, the number of DFTs  is $9$. In this case the DFTs can be performed each time we progress in the buffer of $100$ seconds. After 5 times the DFT duration, a weighted averaging is performed in each filter bandwidth to obtain the SCPs taking into account the cells whose the MSC is above a given threshold, typically $0.98$. The frequencies inside each bandwidth is uniformly spaced.}{}{1}


 %==========================================
%===========================================
%===========================================
 \newpage\clearpage
\section{Objective}
%===========================================
The calibration consists to estimate the impulse response of the SUT or equivalently its frequency response. In the absence of {\it a priori} knowledges, i.e. non parametrical approach, the impulse response is defined by a sequence of values whose the number is related to the bounds of the frequency bandwidth. More specifically, in infrasonic system the lowest frequency is $0.02$ Hz, corresponding to periodicities around $50$ s. On the other hand the largest frequency leads to choose the sampling frequency to $20$ Hz. Therefore, in the absence of {\it a priori} knowledges, the impulse response consists of $50\times 20=1000$ real valued points. Hence the frequency response consists of $500$ complex valued points. However these values are obtained via statistical estimators and therefore several sequences  are required to reach a good accuracy.

There are basically two ways to estimate a linear filter either in time domain or in frequency domain. In both cases to get a good accuracy it is necessary to do averaging. The stochastic description is well-suited for that. The averaging can be applied using blocks of data or by adaptive approach (as Kalman filter or recursive least square). In the adaptive approach we adjust progressively the parameters of interest. This approach has been investigated leading to no interesting results. The main reason is the large variability of the observation. Indeed this kind of approach needs slow evolution of the stationarity. Numerical studies have shown that the best results are obtained if we take into account a small number of observation blocks and omit all the others.

In summary, in the following we only consider that the averaging is performed by block and we have adopted a frequency analysis. Basic tools for that is the spectral representation of wide sense stationary process (see annex \ref{ann:wss}).




%===========================================
\subsubsection{Remark on the deterministic approach (more details page \pageref{ann:deterministic-approach})}
Considering only time intervals with almost zero noises, we can ask: why do we use stochastic approach. Indeed the simple ratio of the short time Fourier transforms of the two observations, where the term short time is related to the long period, is the searched response. But there is also an equivalent to the MSC test. In the deterministic approach we have to validate the assumption that there is no noise and that in some way by locking if the performed ratio is almost constant along successive windows. That can be expressed by saying that the two sequences of Fourier along a few windows are correlated, and that leads to the MSC-like test.


%===========================================
\section{Observation model}
%===========================================
The notations are reported figure \ref{fig:general-schema}.  
%=================================
\subsection{Continuous time model}
%=================================

In the introduction discussion, we have explained that the model of signals are given by expressions \eqref{eq:model-signal-introduction}, which are re-written below:
\begin{eqnarray}
\label{eq:model-of-excitation}
\left\{
\renewcommand\arraystretch{1.6}
\begin{array}{rcl}
e_{\ut}(t)&=&s(t)+w_{\ut}(t)
\\
e_{\rf}(t)&=&s(t)+w_{\rf}(t)
\end{array}
\right.
\end{eqnarray}


The common signal $s(t)$ is the key of the estimation. On the other hand the noises induce a loss of identifiability and appear as a nuisance factor which can be characterized by a loss of coherence (LOC). \\*


It follows that the observation signals write:
\begin{eqnarray}
\label{eq:model-of-obervation}
\left\{
\renewcommand\arraystretch{1.6}
\begin{array}{rcl}
x_{\ut}(t)&=&g_{\ut} (t)\star(s(t)+w_{\ut}(t))
\\
x_{\rf}(t)&=&g_{\rf}(t) \star (s(t)+w_{\rf}(t))
\end{array}
\right.
\end{eqnarray}
%================ figure ====================
\figscale{schema-of-observation.pdf}{Basic observation schema. $\hat S_{UU}(f)$, $\hat S_{RR}(f)$ and $\hat S_{UR}(f)$ represent the three estimated spectral components under stationary assumptions}{fig:general-schema}{0.6}

The stationarity and the absence of correlation between $s(t)$, $w_{\ut}(t)$ and $w_{\rf}(t)$ play a fundamental  role in the identifiability of the response of the SUT. We assume that
\begin{itemize}
\item
$s(t)$ is a wide-sense stationary process with zero-mean and spectral density $\gamma_{s}(f)$,
which is the Fourier transform of the covariance fonction:
\begin{eqnarray*}
 R(\tau) = \esp{s(t+\tau)s(t)}&\rightleftharpoons  &
 S(f)=\int R(\tau) e^{-2j\pi f\tau}d\tau
\end{eqnarray*}
\item
$w_{u}(t)$ and $w_{r}(t)$ are two  stationary processes with zero-mean and respective spectral densities $\gamma_{\ut}(f)$ and $\gamma_{\rf}(f)$. 
\item
$s(t)$,  $w_{\ut}(t)$ and $w_{\rf}(t)$ are jointly independent. More specifically we have for any couple of time $(t,t')$:
\begin{eqnarray*}
\begin{array}{lccr}
 &\esp{w_{\ut}(t)w_{r}(t')}&=&0
 \\
& \esp{w_{\ut}(t)s(t')}&=&0
 \\
& \esp{w_{\rf}(t)s(t')}&=&0
\end{array}
 \end{eqnarray*}
\item
$G_{u}(f)$ represents the frequency response of the SUT which is the combination of the sensor under test, the frequency response of the anti-noise pipes and the digitalizer. $G_{u}(f)$ is the parameter to estimate.
\item
$G_{r}(f)$ represents the frequency response of the SREF. $G_{r}(f)$ is known and has been checked just before the installation.
\end{itemize}

Under these assumptions, the spectral content is fully described by the three components saying the  auto-spectrum of the signal of the SUT, denoted $S_{UU}(f)$, the auto-spectrum of the signal of the SREF, denoted  $S_{RR}(f)$ and the  cross-spectrum, denoted $S_{UR}(f)$. They are related to the input signals and the responses of the sensors by the three following expressions:
\begin{eqnarray}
\label{eq:spectral-model}
\left\{
\renewcommand\arraystretch{1.6}
\begin{array}{rcl}
S_{UU}(f)&=&|G_{\ut}(f)|^{2} (\gamma_{s}(f)+\gamma_{\ut}(f))
\\
S_{RR}(f)&=&|G_{\rf}(f)|^{2} (\gamma_{s}(f)+\gamma_{\rf}(f))
\\
S_{UR}(f)&=&G_{\ut}(f)G^{*}_{\rf}(f)\gamma_{s}(f)
\end{array}
\right.
\end{eqnarray}
It is worth to notice that the cross-spectrum does not depend on the noises because the assumptions of non correlation.


%=================================
\subsection{Discrete time model}
%=================================
We consider that the frequency band of interest is $(1/\tau_{c}, F_{M})$. $F_{M}$ allows to replace the continuous time signals by a time series with a sampling frequency $F_{s}=2F_{M}$. In infrasonic $F_{s}=20$ Hz. $\tau_{c}$ implies to estimate the response up to this period. In the infrasonic context $\tau_{c}=50$ seconds. That implies to reach in the  frequency domain the frequency  $1/\tau_{c}$. Taking a regular grid on the unit circle, as it is with FFT algorithm, the number $L$ of frequency bins will be of an order of magnitude of $\tau_{c}F_{s}$.That leads to take $L$ around $1000$. However in the following numerical studies we have investigated the response up to $500$ seconds, leading to FFT length of $10,\!000$.

 In the following the index $k$ refers to the frequency $kF_{s}/L$. For example $\gamma_{s,k}=\gamma_{s}(kF_{s}/L)$. Because the signal is real, all spectral representations have hermitian symmetry and it is only necessary to consider the positive part of the frequency band. Finally the spectrum at frequency 0 is real valued and will be omitted. Therefore the spectral representation is restricted to the values of the frequency index $k$ going from $1$ to $K=L/2$. \\*

From the expressions \eqref{eq:spectral-model} we obtain for the discrete Fourier transform of the spectral components:
\begin{eqnarray}
\label{eq:spectral-model-discrete}
\left\{
\renewcommand\arraystretch{1.6}
\begin{array}{rcl}
S_{UU,k}&=&|G_{\ut,k}|^{2} (\gamma_{s,k}+\gamma_{\ut,k})\geq 0
\\
S_{RR,k}&=&|G_{\rf,k}|^{2} (\gamma_{s,k}+\gamma_{\rf,k})\geq 0
\\
S_{UR,k}&=&G_{\ut,k}G^{*}_{\rf,k}\gamma_{s,k}\in\mathbb{C}
\end{array}
\right.
\end{eqnarray}
and from the expression \eqref{eq:MSC-continuous-frequency} the MSC expression in the discrete frequency domain:
\begin{eqnarray}
 \label{eq:exact-MSC}
\MSC_{k} = \frac{|{S_{UR,k}}|^{2}}{{S_{UU,k}}{S_{RR,k}}}
\end{eqnarray}

%Also for $\MSC_{k}$ very close to $1$ we can use any of the two formulas \eqref{eq:ratio-sup} or \eqref{eq:ratio-inf}, to estimate the frequency response in the discrete frequency domain. 







%===========================================
%===========================================
%===========================================
\subsection{Resolution of \eqref{eq:spectral-model-discrete} w.r.t. $G_{\ut,k}$}
%===========================================

It is worth to notice tat the last equation of the expression \eqref{eq:spectral-model-discrete} leads to the following expression of the phase of $G_{\ut,k}$:
\begin{eqnarray}
 \label{eq:phase-estimation}
 \arg G_{\ut,k}&=& \arg G_{\rf,k}- \arg S_{UR,k}
\end{eqnarray}

For the module of $G_{\ut,k}$, the problem is a little bit more complicate. Indeed the problem is under-determined in the sense where an infinity of solutions exists. 

\subsubsection{Case of known noise level ratio}
%=================================
In the particular case where we know the ratio between $\gamma_{u,k}$ and  $\gamma_{r,k}$, the equations \eqref{eq:spectral-model} can be solved w.r.t. to $G_{u,k}$. Denoting $\rho_{k}=\gamma_{r,k}/\gamma_{u,k}$ we easily derive:
\begin{eqnarray}
\label{eq:known-noise-ratio}
&&\hspace{-1cm}|G_{\ut,k}|=
\frac{1}{2}\,\sqrt{\frac{S_{UU,k}}{S_{RR,k}}}\,|G_{\rf, k}|\times 
\\&&\nonumber
\left(
(1-\rho_{k})\sqrt{\MSC_{k}}+\sqrt{4\rho_{k}+(1-\rho_{k})^{2}\MSC_{k}}
\right)
\end{eqnarray}
Typically $\rho_{k}$ could be considered as equal to the number of inlets in the noise reduction system, e.g. $96$.




%===========================================
\subsubsection{Under-determination}
In the equations \eqref{eq:spectral-model} the response $G_{\rf,k}$ is perfectly known, the unknows are $G_{\ut,k}\in\mathds{C}$, $\gamma_{s,k}\in\mathds{R}^{+}$,  $\gamma_{\ut,k}\in\mathds{R}^{+}$
$\gamma_{\rf,k}\in\mathds{R}^{+}$. 
 
It follows that, in absence of {\it a priori} knowledges,  the system \eqref{eq:spectral-model} is under-determined with $1$ degree of freedom (d.o.f.). That means that for all $k$ we can, for example,  choose arbitrarily the ratio $\rho_{k}=\gamma_{\ut,k}/\gamma_{\rf,k}$ from $0$ to infinity, then solve the systems \eqref{eq:spectral-model}. If one of the two noises is zero, i.e. the noise ratio is either 0 or infinite, the solution becomes also unique and is given by:





\begin{itemize}
\item
if $\gamma_{\ut,k}=0$, i.e. $\rho_{k}=0$
\begin{eqnarray}
\label{eq:ratio-sup-bis-exact}
{G_{\ut,k}}&=&R_{k}^{\sup}\,G_{\rf,k}, \quad \mathrm{with}\quad
R_{k}^{\sup} = \frac{{S_{UU,k}}}{{S^*_{UR,k}}}
\end{eqnarray}
\item
if $\gamma_{\rf,k}=0$, , i.e. $\rho_{k}=\infty$
\begin{eqnarray}
\label{eq:ratio-inf-bis-exact}
{G_{\ut,k}}&=&R_{k}^{\inf}\,G_{\rf,k}, \quad \mathrm{with}\quad
R_{k}^{\inf}=\frac{{S_{UR,k}}}{{S_{RR,k}}}
\end{eqnarray}
It is easy to show that $|R_{k}^{\inf}|\leq |R_{k}^{\sup}|$, using that the MSC is less than $1$.\\*
\end{itemize}

 Unfortunately there is no way to test if one of the two noises is zero. However it is possible to test that the two noises $w_{\ut,k}$ and $w_{\rf,k}$ are almost negligible. In this case the two estimators given by \eqref{eq:ratio-sup-bis-exact} and \eqref{eq:ratio-inf-bis-exact} are almost equal and almost unbiased. 

The assumptions of negligible noises can be tested by thresholding the Magnitude Square Coherence (MSC) defined by:
\begin{eqnarray}
 \label{eq:MSC-continuous-frequency}
 \MSC_{k} &=& \frac{|S_{UR,}|^{2}}{S_{UU,k}S_{RR,k}}
\end{eqnarray}
 
In the model \eqref{eq:spectral-model} MSC also writes:
\begin{eqnarray}
\label{eq:coherence-in-our-model}
 \MSC_{k} &=& \frac{1}{(1+\iSNR_{\ut,k})(1+\iSNR_{\rf,k})}
\end{eqnarray}
where $\iSNR_{\rf,k}=\gamma_{\rf,k}/\gamma_{s,k}$ and $\iSNR_{\ut,k}=\gamma_{\ut,k}/\gamma_{s,k}$. It follows that the MSC is 1, iff the two noises are zero. We have to keep in mind:
\begin{itemize}
\item
that these equations assume stationarity,
\item
that $G_{\ut,k}$ is performed during the period of time where the two noises are sufficiently low,
\item
that we have no access to the true values of the spectral components but only estimates, inducing dispersion. 
\end{itemize}



%===========================================
%===========================================
%===========================================
\section{Spectral analysis}
%===========================================
The classical moment method leads to replace in the  formulas \eqref{eq:exact-MSC}, \eqref{eq:phase-estimation}, \eqref{eq:ratio-sup-bis-exact}  and \eqref{eq:ratio-inf-bis-exact} the true spectral components by consistent estimates. In the absence of {\it a priori} knowledges that is given by a non parametrical spectral analysis. More details about the statistics of these quantities, as  the ratio of spectral components or the MSC, are presented in annexe \ref{ann:spectral-estimation}. 


%=================================
%===========================================

Based on that, we replace the spectral components $S_{UU,k}$, $S_{RR,k}$ and $S_{UR,k}$ by their consistent estimates $\widehat{S_{UU,k}}$, $\widehat{S_{RR,k}}$ and $\widehat{S_{UR,k}}$, obtained via a Welch's approach by averaging successive periodograms, with a typical overlap of $50\%$ and Hann's window. It is worth to notice that the number of frequency  bins is related to the long period signals whereas the accuracy of the estimates are related to the number of windows which are used for averaging the periodograms, typically we have chosen $5$ windows. 


Also we can find in the annex \ref{ann:spectral-estimation} the details to calculate the probability distributions of $\hMSC_{k}$ and of the ratios $\widehat{S_{UU,k}}/|\widehat{S_{UR,k}|}$ and $|\widehat{S_{UR,k}}|/\widehat{S_{RR,k}}$. We also provide Matlab programs that performs these distributions. 


Remark: it is worth to notice that, in any case, the both following inequalities are satisfied:
\begin{eqnarray}
 |R_{k}^{\inf}|&\leq &|R_{k}^{\sup}|
 \\
  |\widehat{R}_{k}^{\inf}|&\leq &|\widehat{R}_{k}^{\sup}|
\end{eqnarray}
but we have no information about the rank of these four values. It means that the true values can be outside of the estimated ones. Moreover the true values can be also outside of the expectation of the two r.v. $\widehat{R}_{k}^{\inf}$ and $\widehat{R}_{k}^{\sup}$. However when the noises vanish the discrepancy goes to zero.

%=================================
%=================================
\subsection{MSC test}
%=================================

It is worth noticing that we have only an estimate of the MSC not the true value. Therefore a test function has been determined to ensure with a given confidence level, typically $95\%$, to be over a target-value. The target-value is related to the accuracy we want on the estimation of the response of the SUT. In this case an important parameter is the  stationarity duration of the signals. For example we see on figure \ref{fig:allHest} that with $5$ windows, i.e. about 4 minutes (for long period of 50 seconds), the MSC must be over $0.97$ to ensure an RMSE of 5\% on the amplitude response. For such requirements the MSC threshold of the test is about $0.99$, see figure \ref{fig:MSCtestthreshold}.

It is worth to notice that the approach is strongly conservative. Indeed we keep only a few periods of signals, but that could be largely enough if we considered that the calibration has to be done once a year.


\figscale{allHest.pdf}{Simulation: for simulating the sensors are two different IIR(2,2) with important resonances are used. The commun coherent signal $s(t)$ is white. The length of the frequency analysis is $300$ seconds, i.e. $6000$ points. The spectral components are performed on $M$ times this duration. The root mean square error is performed by integrating on the full band. These curves have been obtained with the program {\tt estimHanalysis}.}{fig:allHest}{0.8}

\figscale{MSCtestthreshold.pdf}{Cumulative function of $\hMSC$ for different values of the number $M$ of averaging under the hypothesis that the MSC is $0.96$. These curves provide the threshold to test the hypothesis $H_{0}=\{\MSC>0.96\}$  with a confidence level of $90\%$.}{fig:MSCtestthreshold}{0.8}


  \newpage
 It is important to remark that an RMSE of 5\% means that the measurement has only a probability of $0.7$ to be in this interval, if we assume gaussian asymptotic behavior. Therefore we could expect that we consider at first an accuracy of $10\%$ and reduce in second step this number by aggregating many measurements. The problem is when the MSC is far from $1$ a large indetermination appears which does not leads to a gaussian asymptotic behavior \emph{around the true value}. That is reported on the theoretical curves of the figure \ref{fig:theoreticaldistribratios}. We see that the two ratios are distributed to a gaussian distribution but at a wrong location. It is not possible to correct this bias because that would assume that  the noise ratio is exactly know.


\figscale{theoreticaldistribratios.pdf}{Asymptotic distributions of the two ratios present in the formulas \eqref{eq:ratio-sup} and \eqref{eq:ratio-inf}. These distributions depend on the $\MSC$, but also on the noise levels and that is not known in practical case. These curves are obtained with the program {\tt CIHestimate.m}. The expressions are proved in the annex (see also the provided toolbox).}{fig:theoreticaldistribratios}{0.7}


%=================================
%=================================
\subsection{SUT estimation}
%=================================
In each time window with an MSC above the selected threshold, we can use any of the formulas \eqref{eq:ratio-sup-bis-exact} or  \eqref{eq:ratio-inf-bis-exact} replacing the true values by estimated values. The estimated values are distributed as it is reported figure \ref{fig:theoreticaldistribratios}.  The theoretical expression of the distributions of both ratios are given in annex \ref{ann:spectral-estimation}. In first approximation, we can assume for large values of MSC that the bias is zero and only stays the variance.

Therefore if we assume that the different values of the ratios along a large period of time are identically distributed and statistically independent, we can improve the accuracy by averaging. Also using the level of confidence of each ratio, i.e. its variance which is related to the  estimated MSC, we can perform a weighted average with weights in the inverse of this variance following:
\begin{eqnarray}
\label{eq:weithted-average-Ratio}
 \hat R &=& \frac{1}{L}\sum_{\ell=1}^{L}w_{\ell}\hat R_{\ell}
\end{eqnarray}
where $L$ denotes the number of periods of time with MSC above the selected threshold,
\begin{eqnarray}
\label{eq:estimated-Ratio}
\hat R_{\ell,k} ^{\sup}=\frac{\hat S_{UU,k}^{(\ell)}}{\hat S^{*(\ell)}_{UR,k}}
\quad
\mathrm{or}
\quad
\hat R_{\ell,k}^{\inf} =\frac{\hat S_{UR,k}^{(\ell)}}{\hat S_{RR,k}^{(\ell)}}
\end{eqnarray}
and where $w_{\ell}$ equals the inverse of the variances, whose approximate values are given by \eqref{eq:var12on22} and \eqref{eq:var11on21} and replacing true values by estimated values:
\begin{eqnarray}
\label{eq:weights}
\mathrm{for}\,\, R^{\sup}:  &&1/w_{\ell}=\frac{1}{2(2M+1)}
 \frac{\hat S_{UU,k}^{(\ell)}}{\hat S_{RR,k}^{(\ell)}} 
  \frac{1-\hat\MSC_{\ell}}{\hat\MSC_{\ell}^{2}}
\quad\mathrm{and}\\
\mathrm{for }\,\, R^{\inf}:  && 1/w_{\ell}= \frac{1}{2(2M+1)}
   \frac{\hat S_{UU,k}^{(\ell)}}{\hat S_{RR,k}^{(\ell)}} (1-\hat\MSC_{\ell})
\end{eqnarray}

That is implemented in the fonction {\tt fbankanalysis.m}, look at the flag {\tt weightingflag}.

\subsubsection{Remark}

It is worth to notice that this weighted average is a little bit optimist because based on the assumption that the distribution is identical and the estimates independent.
We might be tempted to use this assumption to identify the parameter of interest as it follows: we start from the equation \eqref{eq:spectral-model-discrete}, labeled by $\ell$ for different observed periods:
\begin{eqnarray}
\label{eq:spectral-model-discrete-ell}
\left\{
\renewcommand\arraystretch{1.6}
\begin{array}{rcl}
S_{UU,k}^{(\ell)}&=&|G_{\ut,k}|^{2} (\gamma_{s,k}+\gamma_{\ut,k})
\\
S_{RR,k}^{(\ell)}&=&|G_{\rf,k}|^{2} (\gamma_{s,k}+\gamma_{\rf,k})
\\
S_{UR,k}^{(\ell)}&=&G_{\ut,k}G^{*}_{\rf,k}\gamma_{s,k}
\end{array}
\right.
\end{eqnarray}
Under the assumption that the unlabeled unknown, saying $\gamma_{s,k}$, $\gamma_{\ut,k}$  and
$\gamma_{\rf,k}$ are identical, we can estimate these values. But it is likely that this approach is not fruitful. Indeed if we look at the distribution of  $R_{k}^{\sup}$ and $R_{k}^{\inf}$  as reported figure \ref{fig:practicalratiodistribution}, it is different from the theoretical distribution given figure \ref{fig:theoreticaldistribratios}. The  main reason is that we mixed many different values of the MSCs.

\figscale{practicalratiodistribution6.pdf}{Left: $R_{\inf}$, right: $R_{\sup}$. The values are obtained in the band of interest of the filter 3. The selected MSC threshold is $0.95$.}{fig:practicalratiodistribution}{0.8}


%================================
%================================
\section{Using a filter bank}
%================================
We had said that the spectral approach is based on stationarity property. But the real signals do not present permanent stationarity. Therefore we have to use time windows where this stationarity can be verified. Moreover is it maybe possible that in the high frequency range, saying e.g. around $1$ Hz, the time window length could be chosen shorter than in the low frequency range, saying e.g. around $0.02$ Hz. It is the main reason to propose a filter bank process in the full processing pipeline. The general pipeline proposed for the estimation process consists:
\begin{itemize}
\item
of a filter bank described in a file of settings which is characterized by a sequence of following descriptors: type, frequency lower bound, upper frequency bound, order, desired stationarity duration etc. Commonly used type is Butterworth (available on Matlab).
\item
of a spectral estimation process. At the output of a filter, the two signals (one for the SUT the other for the SREF) are shared in non-overlapping blocks to perform the spectral estimates. Longer the size of a block more accurate the spectral analysis. But that assumes stationarity, and the real signals do not exhibit permanent stationarity. Therefore we can choose in the setting file the desired stationarity duration in term of multiple of the longest period in the band (inverse of lower frequency bound). Typically we take  $5$ times the longest period, expecting that we can find such length with stationarity to estimate the spectral components. Even with that, the accuracy is not in accordance with the PTS requirements, but by averaging in a long period of times (many days), we can reach such requirements.
\item
Each block is then shared in overlapping sub-blocks. We can choose the windowing and the overlap rate. Typically we consider Hann windowing and $50\%$ of overlapping. The periodograms in the different sub-blocks are averaged to provide a spectral estimation. 

Only estimates in the bandwidth of the filter are retained.
\item
MSC is performed in each bandwidth. If the MSC is above the selected threshold, the value of the ratio $\hat S_{UU,k}/\hat S_{RU,k}$ is saved.
\item
along the recorded data, for a given frequency index $k$, the values  $\hat S_{UU,k}/\hat S_{RU,k}$ are averaged with weighting factors in accordance with $\hMSC$ values.
\item
Finally the estimation of the SUT response is derived from the SREF response.
\end{itemize}

%================================
%================================
\section{Summary for SUT estimation}
%================================
\begin{itemize}
\item
SCP: for spectral components
\item
DFT for discrete Fourier transform (usually computed by FFT). The DFT consists of as many input points and output points. Periodogram refers to the magnitude square of the DFT.
\end{itemize}

In the provided function {\tt fbankanalysis.m} (see annex), from the input values {\tt SCPperiod\_sec} and {\tt ratioDFT2SCP}, we compute the length of the DFT which is the ratio  {\tt SCPperiod\_sec}$/${\tt ratioDFT2SCP}. Using the {\tt overlapDFT} we derive the number of DFTs is needed for computing the SCP.

For example if {\tt SCPperiod\_sec} $=1,\!000$ seconds, {\tt overlapDFT} $= 0.5$ and  {\tt ratioDFT2SCP} $=5$, the DFTs are computed on $500/5=200$ seconds, then for a sampling frequency of 20 Hz, on $4,\!000$ samples. Because the overlap is 0.5 we shift  of $100$ seconds between 2 successive DFTs. It is worth to notice that the resolution is related to the inverse of the DFT time window length, in our example about $1/200=0.005$ Hz.

In the provided function {\tt fbankanalysis.m}, we then compute all DFTs for the full input signals. For example if the full duration  of signals is $3,\!600\times 24$ seconds, we have  $(3,\!600 \times 24)/100=864$ DFTs to compute. Then to compute the SCPs with an overlap {\tt overlapDFT} $=0$, we move on RHS of $9$ DFTs. if {\tt overlapDFT} $=1/10$, we move on $8$ DFTs. That means that the possible overlap for the SCPs are only on the block frontier of the DFTs which has no practical effect.

In practice we advise:
\begin{itemize}
\item
{\tt SCPperiod\_sec} depending of the frequency band to analyze (see section \ref{sss:filterbank}). Lower the analysis frequency band, greater the {\tt SCPperiod\_sec}. But greater the  {\tt SCPperiod\_sec} more difficult could be the probability to find a almost stationary time interval.
\item
{\tt ratioDFT2SCP} $=5$. It is worth to notice that if we increase {\tt ratioDFT2SCP}, for fixed  {\tt SCPperiod\_sec} we reduce the DFT time window length and therefore the resolution. We found empirically that  {\tt ratioDFT2SCP} $=5$ is a good value in terms of compromise resolution/accuracy.
\item
{\tt overlapDFT} $= 0.5$
\item
{\tt overlapSCP} $= 0$
\end{itemize}

\figscale{overlapFFTs.pdf}{Spectral estimation with $\alpha=50\%$ overlapping for FFT block. The variance of the estimate is related to the number of windows. If the number of disjoint FFT block is $M=5$, and if the length in seconds of the spectral analysis window is 500 seconds, the length in second of the time window for 1 FFT is 100 seconds.
The frequency resolution is related to the length of the FFT which is $100$ seconds, hence $0.01$ Hz. The window shape is related to the leakage which is defined as the effect of transferring energy from the bands where the energy is high to the bands where the energy is low. 
The overlapping for successive SCP analysis is $\beta=0$.}{}{0.8}


%=================================
\subsubsection{Filter bank analysis}
\label{sss:filterbank}
%=================================
Let us recall that in a first step, the signals are filtered in adjacent bands in such a way to use different periods whose the main interest is to consider short periods, if necessary, in the high frequency bands. The table \ref{tab:freq-duration-tradeoff} is an example of pavement, consisting of $5$ bands with log-spaced filter parameters in the band $(0.01-5)$ Hz with a variable window length\footnote{See also recent PMCC reports.}. Because the two signals are applied to the same filter we can use RII as Butterworth filter. Also because this process is for anlysis, we don't need to downsampling and/or to require perfect recontruction. Even more the bandpass filters can be overlap in the frequency domain. Although a decimation can be used to save computational time, indeed the bandwidths are lower than $F_{s}/2$, that operation is not considering in the following.

\begin{table}
\begin{center}
\begin{tabular}{|c|c|}
\hline
frequency band (Hz) & stationarity period (second)
\\
\hline
%%%%% from matlab
$[0.02-0.20]$&$400$
\\ \hline $[0.20-1.00]$&$200$
\\ \hline $[1.00-2.00]$&$100$
\\ \hline $[2.00-3.00]$&$50$
\\ \hline $[3.00-4.00]$&$25$
\\ \hline $[4.00-6.00]$&$25$
\\ \hline 
%%%%
\end{tabular}
\parbox{12 cm}
{
    \caption{\protect\small\it  }
    \label{tab:freq-duration-tradeoff}
}
\end{center}
\end{table}
\figscale{filterbank.pdf}{Filter bank}{fig:filterbank}{1}





%===========================================
\chapter{Wind effect on NRS}
\label{chap:windonNRS}
\input{textes/ch5windonNRS.tex}
%===========================================
%===========================================
%===========================================
%============ part 2 ==========================
%===========================================
\part{Numerical results}
%===========================================
\chapter{Full process}
\label{chap:fullprocess}
% !TEX root = ../calibreport.tex


%================================
\section{3 programs for test}
The full process consists of
\begin{enumerate}
\item
 move data from the IDC, using {\tt RUNextractfromDB.m}. The data with Matlab format is saved. This data consists of a structure called {\tt records} with the following items:
 \begin{itemize}
 \item
{\tt data}: sequence of samples,
 \item
{\tt Fs\_Hz}: sampling frequency,
\item
{\tt stime}, {\tt time}: start and end times, 
\item
{\tt station}: I26C1--I26C8 or I26H1--I26H8
\item
{\tt channel}: 'BDF' or 'LKO' or 'LWD' or 'LWS'
 \end{itemize}   
Typically in the following we use BDF on H and C.   The three other channels are only provided on H1. Channel LWD yields  the wind speed measurement. The wind sensor is located at about 2 meters above the infrasonic sensor. The frequency sampling depends on the channel. Typically for infrasound signals, the frequency sampling is $20$ Hz, and for wind features is 1 Hz.
    
\item
 using the filtercharacteristics, saved in {\tt filtercharacteristics.mat} and the directory which contains the data from one of the 8 channels, run the program {\tt estimationwithFB.m}. This program saves the useful elements for a display in a selected directory.
 \item
 to display run the program {\tt displaySUTresponse.m} for the file recorded in the selected directory of the previous item.
\end{enumerate}


%================================
%================================
%================================
\section{Input of the function {\tt fbankanalysis.m}}
\begin{itemize}
\item
signals: an array of size $N\times 2$ where $N$ is a number of samples, the first column is the signal from the SUT channel and the second for the SREF channel,
\item
filter bank settings: see section \ref{sss:filter-bank-settings},
\item
$F_s$: sampling frequency in Hz, typically $20$ Hz,
\item
MSC threshold: threshold for the MSC, typically $0.98$
\end{itemize}

%================================
\subsubsection{Summary of the filter bank settings}

\label{sss:filter-bank-settings}

\begin{center}
{\tiny\verbatiminput{filtercharacteristicsEXAMPLE.m}}
\end{center}

\begin{itemize}
\item
{\tt designname} means that the name of the filter model. The Butterworth model is available in Matlab. 
\item
{\tt Norder} denotes the order of the filter. If 0 there is no filtering. 
\item
{\tt Wlow\_Hz} denotes the low bound in Hz of the filter design,
\item
{\tt Whigh\_Hz} denotes the high bound in Hz of the filter design,
\item
{\tt windowshape} denotes the weighted window. Many windows are available in Matlab. Window is used to do a compromise between the leakage and the bias. Hann's window is commonly used.
\item
{\tt SCPperiod\_sec} denotes the time duration expressed in second of the window used for the spectral estimation,
\item
{\tt overlapDFT} overlapping rate for the spectral estimation,
\item
{\tt overlapSCP} overlapping rate for the different spectral estimates,
\item
{\tt ratioDFT2SCP} is an integer which denotes the ratio between the duration of the spectral estimation window and the duration of the DFT window. 

\end{itemize}

\newpage\clearpage

%================================
\subsubsection{Calculation scheme}
\figscale{figures/synoptic.pdf}{FB analysis}{fig:synoptic}{1}

Let us report to the figure \ref{fig:synoptic}. The DFT buffer contains the data for a duration in accordance with the DFT analysis. Usually this duration is in the inverse of the bandwidth of the filter. The filter bandwidth is log-scaled in the frequency domain. If the SCP time window is $1000$ seconds and the DFT buffer duration is  a fifth i.e. $200$ seconds with an overlap of $50\%$, the number of DFTs  is $9$. In this case the DFTs can be performed each time we progress in the buffer of $100$ seconds. After 5 times the DFT duration, a weighted averaging is performed along the time in each filter bandwidth to obtain the SCPs taking into account the cells whose the MSC is above a given threshold, typically $0.98$. The frequencies inside each bandwidth is uniformly spaced. With the current values of the filter bank, the number of frequency values for the full bandwidth estimation is less than $100$. In the matlab function (see section \ref{chap:toolbox}) the averaging is obtained on two successive days.





The full calculation is based on the 3 following functions
\begin{itemize}
\item 
filtering function {\tt fbank.m}: 
\begin{itemize}
\item
the function inputs are the two signals, the filter characteristics and the frequency sampling.
\item
The filter ouputs are the $P$ output signals.
\end{itemize}
\item  

SCP function {\tt estimSCP.m}: the inputs are the two signals, the frequency response of the SREF, 
the DFT overlap, the SCP overlap, the frequency sampling and the smooth window. The outputs are the SCPs which consist of $P$ time-spectral cells, each of them consists of an array of size $K \times T$ where $K$ is the number of frequency dots and $T$ the number of SCP time windows during the full duration of the considered signals.

\end{itemize}



\newpage\clearpage
%================================
\section{Output of the function {\tt fbankanalysis.m}}

\begin{verbatim}
% 1) SUTs: structures P x 1
%         xx.estimRsup: Rsup ratio
%         xx.estimRinf: Rinf ratio
%         xx.allMSCs: allMSCs;
%         xx.Nsupthreshold: count of the number of values over
%                   the threshold
%         xx.Nsupthresholdintheband: counts of the number of values
%                   over the threshold in the filter bandwidth
%         xx.frqsFFT_Hz: cell P x 1, each cell consists of
%                   frequency list in Hz of the DFTs
%         xx.SCP = all spectral components
%         xx.indexinsidefreqband = P x 1, indices of the
%                    frequency bounds of each filter
%                    in the xx.frqsFFT_Hz
% 2) filteredsignals, 
% 3) allfrqsFFT_Hz, cell P x 1, each cell for each bank filter consists of
%             frequency list in Hz of the DFTs
% 4) alltimes_sec: cell  Px 1, each cell consists of the structure
%             yy.FFT:     time list in second of the DFTs
%             yy.SD:      time list in second of the SCPs
%             yy.signals: time list in second of the signals
% 5) filterbank



\end{verbatim}


%================================
\subsubsection{Remark for developper}

It is advised to provide at the output of the FB analysis function, on one hand the response ratios in each of the $P$ frequency bands  with the associated period and on the other the MSC levels greater than the given threeshold. These two elements have the same size. An external function will perform the averaging over a given period of time with the weights as defined in expression \ref{eq:weights}.

It follows that 
\begin{itemize}
\item
 an extra input must be considered to take into account the duration of the averaging. In our code this time is fixed to two consecutive days, i.e. $48$ hours
\item
 an extra output must be considered to provide the MSC levels associated to each spectral component. We only consider the time-frequency cells where the current MSC is above the given threshold.

\end{itemize}


%===========================================
\chapter{Results of IS26 - sensor \#1}
\label{chap:IS26results}
% !TEX root = ../calibreport.tex
%================================
The results of this section reports the estimation for the eight SUT given in the table \ref{tab:sensor-specifications}.
The analysis is on the data from IS26. Each query on the DB is on 2 consecutive days from the 1st of June to the 9th of October, saying about 130 days. For each file a spectral analysis is processed leading to an SUT response estimate. From the spectral components we perform, in each frequency band and for each time interval, the response of the system under test (SUT) using the known response of the reference sensor. These responses includes both the sensor and the digitizer. These responses are averaged over the full set of time intervals over the all days. The  MSC threshold is fixed to $0.98$. 

Let us remain the PTS specifications:
\begin{itemize}
\item
IMS pass-band requirement $[0.02 - 4]$ Hz;
 \item
$\pm 5\%$ on the response magnitude, i.e. $\pm 0.43$ in dB scale;
\item
the calibration is required at least once a year;
 \item
no requirement  on the phase but $\ldots$ $\pm 5\degree$ as for seismic requirements.
\end{itemize}


 \newpage
%================================
\section{Results of IS26 - sensor \#1}
%================================
The average estimate of the SUT response, interm of gain in dB and phase in degree, are reported figure \ref{fig:3monthsonIS26SUTboxplot1}.
%================================
\subsection{SUT response in the full band averaged on the full period of time}
%================================
\figscale{slidesITW2015/3monthsonIS26SUTboxplot1.pdf}{The red dashed lines report the theoretical SUT response range requirement, which is $\pm 0.43$ for the gain and $\pm 5\degree$ for the phase.}{fig:3monthsonIS26SUTboxplot1}{0.8}

%================================
\subsection{Full band on two successive days}
%================================
\figscale{slidesITW2015/averagedon20days1.pdf}{Sup ratio given by the expression \eqref{eq:ratio-sup-bis-exact}. The grey dots report the ratios averaged on a pair of 2 consecutive days, the red  disks report the average on 10 pairs of days. }{}{0.8}

%================================
\subsection{Full band for $20$ randomly selected days}
%================================
\figscale{slidesITW2015/averagedon20days1.pdf}{Sup ratio given by the expression \eqref{eq:ratio-sup-bis-exact}. The grey dots report the ratios averaged on a pair of 2 consecutive days, the red  disks report the average on 10 pairs of days. }{}{0.8}

%================================
\newpage\clearpage
%================================
\subsection{Stability at $1$ Hz}
%================================

%================================
\newpage\clearpage
%================================
%================================
\section{Comments}
%================================
\subsection{About the selected MSC threshold}
%================================
We said that we can not choose a too low value for the MSC threshold because the underdetermination. That appears clearly on figure \ref{fig:afewdaysonI26C6H6twoMSC}.
When the  MSC threshold is $0.7$, there is a large discrepancy between the estimated ratios, see \eqref{eq:estimated-Ratio} and we outline that there is no way to solve the underdetermination. But for an MSC threshold of $0.95$, the two curves are very close.

\figscale{afewdaysonI26C6H6twoMSC.pdf}{Couple H6C6, for two MSC thresholds.}{fig:afewdaysonI26C6H6twoMSC}{0.7}

On the other hand if the ratio between the two noise levels are perfectly known the indetermination is removed and we can use formula \eqref{eq:known-noise-ratio}. You might be tempted to say that the two noises on the two sensors are identical except theirs levels and consider that the ratio is given by the number of inlets in the noise reduction system. However the figure \ref{fig:afewdaysonI26C4H4knownnoiseratio} shows that the curve with a MSC threshold of $0.5$ and the formula \eqref{eq:known-noise-ratio} lead to values different from the curves obtained with a a MSC threshold of $0.95$.

\figscale{afewdaysonI26C4H4knownnoiseratio.pdf}{Couple H4C4, for two MSC thresholds with formula \eqref{eq:known-noise-ratio}.}{fig:afewdaysonI26C4H4knownnoiseratio}{0.7}

%================================
\newpage\clearpage
%================================
\subsection{Dip on the curves}
%================================
For the figures \ref{fig:C2} to \ref{fig:C5} associated to the MB2005, an important dip around 0.1 Hz is observed. 

Also we have reported figure \ref{fig:afewdays1colocation} a few days on the location H2C2. The different colors are for different days. The distribution seems to be uniform and does not depend on the day. The ratio seems to be different of 1.\\*

If the MSC is about 0.99, meaning that the noises on the two sensors are negligible, and if the two sensors have the same response, the only reason to get a ratio less than 1, is that the acoustical SOI on the SUT is attenuated, may due to the noise system reduction. That would write: it exists $\alpha\in\mathbb{C}$ with $|\alpha|<1$ s.t.:
\begin{eqnarray}
\label{eq:model-of-obervation}
\left\{
\renewcommand\arraystretch{1.6}
\begin{array}{rcl}
x_{\ut}(t)&=&g_{\ut}  \star (\alpha s(t))
\\
x_{\rf}(t)&=&g_{\rf}  \star s(t)
\end{array}
\right.
\end{eqnarray}

\figscale{afewdays1colocation.pdf}{a few days on H2C2. Only the band $[0.08-0.12]$ Hz is selected. The coherence is above $0.99$. The different colors for the different days.}{fig:afewdays1colocation}{0.5}


Another way to see the response differences between the 2 sensors in the band  $[0.08-0.12]$ Hz (gain ratio different of 1), appears figure \ref{fig:filteredsignals}. A zoom of the signals is plotted. It consists of about 1 minute, around a position where the observed coherence is above $0.99$. We see that the signal on the SREF  is bigger than this on the SUT. There is no way and no reason to reject this time window since the difference can be due either to the loss of gain or a unknown transfer function. 
\begin{figure}%{20cm}
\begin{minipage}{10cm}
              \includegraphics[scale=0.5]{signalsanomaly.pdf}
\end{minipage}
\begin{minipage}[c]{8cm}
              \includegraphics[scale=0.5]{filteranomaly.pdf}    

\end{minipage}
\centering
\caption{Filtered signals}
\label{fig:filteredsignals}
\end{figure}


%===========================================

%===========================================
%===========================================
%===========================================
%============ part 3 ==========================
%===========================================
\part{Annexes}
\chapter{Statistic for mean estimation}

% !TEX root = ../calibreport.tex
%==============================================================
\section{Trimmed mean}
%==============================================================
The classical estimator of the mean writes:
\begin{eqnarray*}
\hat \mu^{(1)}_{N} &=&\frac{1}{N}\sum_{n=1}^{N}X_{n}
\end{eqnarray*}
It is well known that this estimator is sensitive to the presence of outliers. In this case it is well known that the median is more robust. The median writes
\begin{eqnarray*}
\hat \mu^{(2)}_{N} &=&X_{(N/2)}
\end{eqnarray*}
where $X_{(n)}$ denotes the $n$-th value of the ordered sequence.

The trimmed mean is a compromise between the mean and the median. It consists to remove a given percent of the extreme values:
\begin{eqnarray*}
\hat \mu^{(3)}_{N} &=&\frac{1}{\lfloor N(1-2\alpha)\rfloor}\sum_{n=\lfloor \alpha N \rfloor }^{\lfloor N((1-\alpha)\rfloor }X_{n}
\end{eqnarray*}
where $\alpha$ is between $0$ and $0.5$

%==============================================================
\section{Confidence interval on the mean}
%==============================================================

We consider a sequence of $N$ data modeled as $N$ i.i.d. r.v. denoted $X_{n}$. An estimator of the mean is given by
\begin{eqnarray*}
\hat \mu_{N} &=&\frac{1}{N}\sum_{n=1}^{N}X_{n}
\end{eqnarray*}
We let
\begin{eqnarray*}
\hat\sigma_{N}^{2} &=& \frac{1}{N-1}\sum_{n=1}^{N}(X_{n}-\hat \mu)^{2}
\end{eqnarray*}
To provide a confidence interval (CI) for $\hat\mu_{N}$ we have
\begin{itemize}
\item
if $X_{n}$ are gaussian with mean $\mu$ and variance $\sigma^{2}$ both unknown, it is show that the CI can be derived from the Student distribution with $N-1$ d.o.f. more specifically we have:
\begin{eqnarray*}
\frac{\sqrt{N}(\hat \mu_{N} -\mu)}{\hat\sigma_{N}}&\sim&T_{N-1}
\end{eqnarray*}
leading to the CI:
\begin{eqnarray*}
\hat\mu_{N}-\alpha\frac{\hat\sigma_{N}}{\sqrt{N}}
&
 \leq\mu\leq
&
\hat\mu_{N}+\alpha\frac{\hat\sigma_{N}}{\sqrt{N}}
\end{eqnarray*}

\item
for large $N$ (limit central theorem), it is shown that 
\begin{eqnarray*}
\frac{\hat\sigma_{N}}{\sqrt{N}}
\end{eqnarray*}
leading to the CI:
\begin{eqnarray*}
\hat\mu_{N}-\beta\frac{\hat\sigma_{N}}{\sqrt{N}}
&
 \leq\mu\leq
&
\hat\mu_{N}+\beta\frac{\hat\sigma_{N}}{\sqrt{N}}
\end{eqnarray*}

\item
for limit central theorem, 
\end{itemize}




%==============================================================
\section{Confidence interval on an estimator}
%==============================================================

%===========================================
\chapter{Wide sense stationary process}
\label{ann:wss}
\input{textes/a1wss.tex}
\chapter{Theoretical results on spectral estimation}
\label{ann:spectral-estimation}
 \input{textes/a2spectralestimationDistribution.tex}
\chapter{LOC}
 % !TEX root = ../calibreport.tex
%============================================

Let us consider the equation:
$$
 Y_{t}=g_{t} \star X_{t}
$$
where $g_{t}$ is the $M$-ary impulse response of the single input multiple output (SIMO) linear filter and $X_{t}$ is a scalar WSS process, with zero mean and spectral matrix (of size 1) $\gamma_{x}(f)$. It is easy to show that the spectral matrix of $Y_{t}$ writes
$$
 \Sigma_{y}(f)=\gamma_{x}(f)G(f)G^{H}(f)
$$
where $G(f)$ is the discrete time Fourier transform of $g_{t}$.
We notice that $\Sigma_{y}(f)$ is a matrix of rank 1. A particular case is a pure delay propagation filter as for example in the planar propagation where the vector entry of $G(f)$ writes:
$$
 G_{m}(f)=e^{-2j\pi f r_{m}^{T}\theta}
$$
where $r_{m}$ is the 3D location of the sensor $m$ and $\theta$ the wave number of the planar wave. It follows that the coherence matrix of the $M$-ary observation has the following entry expression
\begin{eqnarray}
 \label{eq:delaymatrixpuredelay}
 C_{y,m\ell}(f)&=&e^{-2j\pi f (r_{m}-r_{\ell})^{T}\theta}
 \\ \nonumber
 &=&\int_{\mathds{R}^{3}} e^{-2j\pi f (r_{m}-r_{\ell})^{T}u}\delta_{\theta}(u)du
\end{eqnarray}
where $du$ is the infinitesimal Lebesgue measure in $\mathds{R}^{3}$ and $\delta_{\theta}$ the Dirac's distribution located in $\theta$. The expression \eqref{eq:delaymatrixpuredelay} can be seen as an expectation of $e^{-2j\pi f (r_{m}-r_{\ell})^{T}\theta}$ assuming that $\theta$ is deterministic.

%===================================================================
\subsubsection{Coherence model}

In a pioneer work Mack and Flinn \cite{mack_flinn:1971} propose to model the loss of coherence by considering that the azimuth, the elevation and velocity are uncertain. More specifically, authors in \cite{nouvellet:2012}  assume that the 3D vector $\mu=(a,e,c)$ where $a$ denotes the azimuth, $e$ the elevation and $c$ the sound velocity writes:
\begin{equation}
 \label{eq:randomnesswn}
 \mu=\mu_{0}+\nu
\end{equation} 
where $\mu_{0}=(a_{0},e_{0},c_{0})$ is a deterministic value and $\nu$ a zero-mean Gaussian random vector of dimension 3, whose covariance is denoted $\Sigma_{\mu}$.  We know that $\mu$ is related to the slowness vector $k$ by the one-to-one mapping 
\begin{eqnarray*}
 \label{eq:tehtaxyz}
 {\footnotesize
 \begin{array}{rcl}
 f:\,
  \mu= 
 \left\{
 \begin{array}{ll}
 a=\arg(k_{2}+jk_{1})&a\in(0,2\pi)
 \\
 e=\arg\sin(ck_{3})&e\in(-\pi/2,\pi/2)
 \\
 v=(k_{1}^{2}+k_{2}^{2}+k_{3}^{2})^{1/2}&v\in\mathds{R}^{+}
 \end{array}\right.
 &
 \Longleftrightarrow\vspace{6pt}
 \\ 
 &&\hspace{-5.5cm}
 \,k=\left\{
 \begin{array}{ll}
 k_{1}=-v^{-1}\sin(a)\cos(e)
 \\
 k_{2}=v^{-1}\cos(a)\cos(e)&\in\mathds{R}^{3}
 \\
 k_{3}=v^{-1}\sin(e) 
 \end{array}\right.
 \end{array}}
\end{eqnarray*}
Using a first order Taylor's expansion, it follows that $k \approx f(\mu_{0})+J(\mu_{0})(\mu-\mu_{0})$ where the Jacobian writes
\begin{eqnarray*}
 J(\mu_{0}) 
&=&
v_{0}^{-1} \times \tilde J(\mu_0)
\\
\mathrm{with}&&
\tilde J(\mu_0) = 
\begin{bmatrix}
 -\cos(a_{0})\cos(e_{0})&
 \sin(a_{0})\sin(e_{0})&
 v_{0}^{-1}\sin(a_{0})\cos(e_{0})
 \\
 -\sin(a_{0})\cos(e_{0})&
 -\cos(a_{0})\sin(e_{0})&
 -v_{0}^{-1}\cos(a_{0})\cos(e_{0})
 \\
 0& 
 \cos(e_{0})&
 -v_{0}^{-1}\sin(e_{0})
 \end{bmatrix}
\end{eqnarray*}
Therefore from \eqref{eq:randomnesswn} we derive $k\approx k_{0}+\varepsilon$ where $\varepsilon=J(\mu_{0})\,\nu$ appears as a zero-mean Gaussian random vector whose the covariance is:
\begin{eqnarray}
 \label{eq:aec2theta}
\Sigma_{k} \approx v_{0}^{-2} 
\underbrace{\tilde J(\mu_{0}) \Sigma_{\mu} \tilde J^{T}(\mu_{0})}_{\tilde \Sigma_{k}}
\end{eqnarray}
Hence the coherence matrix entry of the model with random wavefront writes:
\begin{eqnarray*}
   C_{m,\ell}(f)&=&e^{-2j\pi f(r_{m}-r_{\ell})^{T}k_{0}}
   \Phi_{\epsilon}(2\pi f(r_{m}-r_{\ell}))
\end{eqnarray*}
where $\Phi_{\epsilon}:u\in\mathds{R}^{3}\mapsto\esp{e^{ju^{T}\varepsilon}}\in\mathds{C}$ is the characteristic function of $\varepsilon$. 
Since $\varepsilon$ is Gaussian distributed with zero-mean and covariance matrix $\Sigma_{k}$, hence we have
\begin{eqnarray}
\label{eq:Cfr1r2Gaussian}
C_{m,\ell}(f)=
\underbrace{e^{-2j\pi f(r_{m}-r_{\ell})^{T}k_{0}}}_{\text{pure delay}}
 \underbrace{e^{-2\pi^{2}(f/v_0)^{2}(r_{m}-r_{\ell})^{T}
                 \tilde \Sigma_{k}(r_{m}-r_{\ell})}}_{\text{LOC}}
 \end{eqnarray}

We let  $\zeta=v_0/f$ that can be interpreted as a ``wavelength''. Using the MSC definition \eqref{eq:def-coherence-function}  we derive the following expression:
\begin{eqnarray}
\label{eq:MSCwind}
\MSC_{km}(f)=
e^{-\frac{4\pi^{2}}{\zeta^{2}}(r_{m}-r_{\ell})^{T}
                 \tilde \Sigma_{k}(r_{m}-r_{\ell})}
 \end{eqnarray}
It follows that
\begin{itemize}
\item
 when $\zeta$ is large w.r.t. the sensor  inter-distances, the matrix $C\approx I$, hence the components of $x(t)$ appear as non spatially coherent,


\item
when $\zeta$ is small w.r.t.  the sensor  inter-distances, the matrix $C\approx \mathds{1} \mathds{1}^{T}$ which is a projector, hence the components of $x(t)$ appear as spatially coherent.
 

\end{itemize}

\chapter{Remark on deterministic representation}
 % !TEX root = ../calibreport.tex

%====== deterministic

\subsubsection{Remark on a deterministic approach}
\label{ann:deterministic-approach}
We have seen that, in the absence of noises, we can use the DFT of each trajectory and perform the ratio without any averaging. But we have also to express a deterministic property which can be the equivalent of the MSC condtion. For that we can work frequency by frequency and therefore omit the dependence in frequency. We denote $X_{\ut,1}$, $\ldots$ 
 $X_{\ut,M}$ the DFT of the signal of the SUT and $X_{\rf,1}$, $\ldots$ 
 $X_{\rf,M}$ the DFT of the signal of the SREF.

Now to define in the determistic case a notion equivalent to the MSC, we can say that the values of the ratio must be almost constant along the $M$ windows. That writes: it exists 2 coefficients   not depending on $m$, verifying $\alpha_{\ut}^2+\alpha_{\rf}^2=1$ and s.t. 
\begin{eqnarray} 
\label{eq:approx}
 \alpha_{\ut}X_{\ut,m}+\alpha_{\rf}X_{\rf,m}&\approx &0
\end{eqnarray}
With obvious matricial notations, we have:
\begin{eqnarray*}
 X \alpha &=& \epsilon
\end{eqnarray*}
\begin{eqnarray*}
X & = &
\begin{bmatrix}
X_{\ut,1}&X_{\rf,1}
\\
\vdots
\\
X_{\ut,M}&X_{\rf,M}
\end{bmatrix} 
\quad\mathrm{and}\quad
\begin{bmatrix} 
\alpha_{\ut}
\\
\vdots
\\
\alpha_{\rf}
\end{bmatrix} 
\end{eqnarray*}
The minimization of the norm of $\epsilon$ under the constraint $\|\alpha\|^2=1$ leads to take for $\alpha$ the eigenvector associated to the lowest eigenvalue of denoted $\lambda_{\min}$. The matrix $M^{-1}X^TX$  writes:
\begin{eqnarray*}
\Gamma&=&\frac{1}{M}
 \begin{bmatrix}
\sum_{m=1}^M |X_{\ut,m}|^2&\sum_{m=1}^M X_{\ut,m}X_{\rf,m}^*
\\
\sum_{m=1}^M X_{\ut,m}^*X_{\rf,m}&\sum_{m=1}^M |X_{\rf,m}|^2
\end{bmatrix}
\end{eqnarray*}

The minimum value is then given by:
\begin{eqnarray*}
 \epsilon_{\min}&=&\lambda_{\min}
\end{eqnarray*}
Lower  $\lambda_{\min}$ better the approximation \eqref{eq:approx}. On the other hand, lower  $\lambda_{\min}$ lower the determinant of $\Gamma$ which writes:
\begin{eqnarray*}
 \Delta & =& M^{-2}\sum_{m=1}^M |X_{\ut,m}|^2\sum_{m=1}^M |X_{\rf,m}|^2-
  M^{-2}\left| \sum_{m=1}^M X_{\ut,m}X_{\rf,m}^* \right|^2
 \\
&=& M^{-2}\sum_{m=1}^M |X_{\ut,m}|^2\sum_{m=1}^M |X_{\rf,m}|^2\,
(1-\MSC)
\end{eqnarray*}
In conclusion nearer the MSC to 1, better the approximation \eqref{eq:approx}.



%===========================================
%===========================================
\part{Library}
 %============================================
\chapter{Programs}
\label{ch:programs}
For practical reasons related to the use of {\tt GIT}, the directory consisting of the signals extracted for the {\tt testbed\_archive} database are outside of the directory {\tt fullprocess}. 

 % !TEX root = ../calibreport.tex
%================================
%============================================ 
\section{Toolbox}
\label{s:toolbox}
{\tiny \verbatiminput{\programsToolbox fbankanalysis.m}}

\section{Cumulative function of $\hMSC$}
{\tiny \verbatiminput{\programsToolbox cumulFunctionMSC.m}

\section{Inverse cumulative function of $\hMSC$}
{\tiny \verbatiminput{\programsToolbox invcumulFunctionMSC.m}

\section{Probability density function of $\hMSC$}
{\tiny \verbatiminput{\programsToolbox pdfMSC.m}

 \section{Statistics of the spectral ratios}
{\tiny \verbatiminput{\programsToolbox theoreticalStats.m}

 \section{Polynomial fitter}
{\tiny \verbatiminput{\programsToolbox smoothpolyLL.m}

%============================================
 \section{Full process}

 \subsection{Example of filter bank description}
{\tiny \verbatiminput{\programsfullprocess filtercharacteristics/filtercharacteristics1.m}}

 \clearpage
\subsection{Main program for estimation}
{\tiny \verbatiminput{\programsfullprocess estimationwithFB.m}}


 \clearpage
%============================================
\section{Application programs}
The following program are used to draw several figures of the report.

\subsection{displaySUTresponse.m}
{\tiny \verbatiminput{\programsfullprocess progs2display/displaySUTresponse.m}}

  \clearpage
\subsection{evaluatetheSTDs.m}
{\tiny \verbatiminput{\programsfullprocess progs2display/evaluatetheSTDs.m}}

 \clearpage
\subsection{temporalevolution.m}
{\tiny \verbatiminput{\programsfullprocess progs2display/temporalevolution.m}}

 \clearpage
\subsection{RsupAsFreqfor3threshold.m}
{\tiny \verbatiminput{\programsfullprocess progs2display/RsupAsFreqfor3threshold.m}}

 \clearpage
\subsection{displayafewproblemsonRsup.m}
{\tiny \verbatiminput{\programsfullprocess progs2display/displayafewproblemsonRsup.m}}

 \clearpage
\subsection{plotsignalwithproblem.m}
{\tiny \verbatiminput{\programsfullprocess progs2display/plotsignalwithproblem.m}}

 \clearpage
\subsection{plotRsupdetails.m}
{\tiny \verbatiminput{\programsfullprocess progs2display/plotRsupdetails.m}}


%============================================ 
\chapter{Extraction form DB}

 \clearpage
\section{RUNextractfromDB}
{\tiny \verbatiminput{\programsfullprocess RUNextractfromDB.m}}

 \clearpage
\section{execGPARSE}
{\tiny \verbatiminput{\programsToolbox 00pierrick/extractfromDB.m}}

 \clearpage
\section{convertCSS2matlab}
{\tiny \verbatiminput{\programsToolbox 00pierrick/convertCSStomatlab.m}}

 \clearpage
\section{savesignals}
{\tiny \verbatiminput{\programsToolbox 00pierrick/savesignals.m}}


%============================================ 
\chapter{Miscellaneous utilities}
\section{GeneFB}
{\tiny \verbatiminput{\programsToolbox geneFB.m}}

 \section{RMSE as function of coherence}
{\tiny \verbatiminput{\programsfullprocess/progsCIandOthers/estimHanalysis.m}}

  \newpage
  \section{Test of coherence}
{\tiny \verbatiminput{\programsfullprocess/progsCIandOthers/SigniLevelTestCoherence.m}}
 
 \newpage
 \section{Statistical distribution of the ratios}
{\tiny \verbatiminput{\programsfullprocess/progsCIandOthers/CIHestimate.m}}

  
\newpage
%============================================
  \section{Extraction from DB}
  
 \subsubsection{Settings for the query to the database}
{\tiny \verbatiminput{\programsfullprocess RUNextractfromDB.m}}

 \subsubsection{Query to the database}
{\tiny \verbatiminput{\programspierrick extractfromDB.m}}

 \subsubsection{Conversion to format {\tt .mat}}
{\tiny \verbatiminput{\programspierrick convertCSStomatlab.m}}




%===========================================
%===========================================
\bibliography{../../../allbib}

\end{document}

%===========================================
%%===========================================
%%===========================================
%\chapter{Simulation}
%\input{textes/ch3simulation.tex}
%%===========================================


%\chapter{Spectral Estimation}
% %====== spectral estimation

\subsubsection{Spectral estimation}
\label{ann:spectral-estimation}
Estimation of the frequency response of the sensor under test needs the estimation of the spectral matrix of the process assumed it is stationary. The main ingredient is the following asymptotic result:  the $K$ 
 frequency points of the discrete Fourier transform (DFT) are independent bivariate gaussian-circular random vectors with zero-mean and covariance $\Gamma_k$. $\Gamma_k$ is a $2\times 2$ matrix depending on $4$ real values. 

In the general case, the statistical model is identifiable and the estimation of $\Gamma_k$ is obtained via the DFT by:
 \begin{eqnarray}
\label{eq:hatGammak}
\widehat\Gamma_k = \sum_k w_M(k)I_N(k), \quad\mathrm{where}\quad
I_N(k) = \frac{1}{N}\sum_{n=0}^{N-1}x_n\,e^{-2i\pi nk/N}
\end{eqnarray}
Unfortunately in our context, the spectral matrix has the following form:
 \begin{eqnarray}
\label{eq:GammakinourModel}
\Gamma_k = \begin{bmatrix}
|G_k|^2(\gamma_k+\sigma^2_{1,k})&G_kH_k^*\gamma_k\\
G_k^*H_k\gamma_k&|H_k|^2(\gamma_k+\sigma^2_{2,k})
\end{bmatrix}
\end{eqnarray}
whose the parametrization is not identifiable, consisting on $G_k\in\mathds{C}$, $\gamma_k\geq 0$, $\sigma^2_{1,k}\geq 0$ and $\sigma^2_{2,k}\geq 0$, i.e. $5$ real parameters whereas $\widehat\Gamma_k$ has only $4$ real values.





%We have reported figure \ref{fig:signalsTF} about 5 minutes of observations from IS26 on the SUT and the SREF and also the spectral contents. It appears that the SREF contains more energy in the medium frequency domain. That is due to the noise reduction system of the SUT, working better in the middle frequencies.

 
%\figscale{signalsTF.pdf}{Two top figures: signals in time domain. Third figure: signals in the frequency domain, performed on 50 seconds DFTs averaged on 500 seconds (10 DFT windows). Forth figure: MSC. All levels are considered up to a multiplicative constant.}{fig:signalsTF}{0.7}

%Figure \ref{fig:MSCovereta} we have reported the time-frequency cells where the MSC is greater than the indicated level. MSC has been computed by averaging DFT over 500 seconds. The figure corresponds to about 1 day. On the bottom figure we observe that we can expect, at any frequency, a large number of time slots where the MSC is close to one.

%\figscale{MSCovereta.pdf}{Top figure: only MSC over 0.98 are reported. The periodograms are averaged over 500 secondst. Bottom figure: counts of number of values of MSC greater than 0.98 observed during one day.}{fig:MSCovereta}{0.8}

%\figscale{withandwithoutonbothestimators.pdf}{Day \#137 in 2015 on IS26. Ratios with/without the constraints for the two expressions \eqref{eq:ratio-sup} and \eqref{eq:ratio-inf}. As expected over the MSC threshold, the two ratios are almost equal. Also without thresholding, the curves seem to indicate that the noise on the SUT is lower than on the noise on the SREF. The tone at $3.68$ Hz is a test signal.}{fig:withandwithoutonbothestimators}{0.7}









