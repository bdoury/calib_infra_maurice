% !TEX root = ../distCjHMSC.tex
%===========================================


In this chapter we derive the expression of the three spectral components associated with the two outputs signals, as a function of the responses of the two sensors. Without \emph{a priori} knowledges  the problem is ill-conditioned. An approach based on the MSC can be used to remove the uncertainty. That leads to an estimation algorithm by replacing the spectral components by consistent estimates. These estimates are obtained via Welch's technic. Finally a filter bank approach is shown to be useful. Indeed spectral estimation  needs that the signals are stationary for long periods of time, which can be  very challenging particularly in the low frequency domain. For circumvent this difficulty we propose to separate the signals in different frequency bands.

 %===========================================
%===========================================
%===========================================
\section{Objective}
%===========================================
The calibration consists to estimate the impulse response of the SUT or equivalently its frequency response. In the absence of {\it a priori} knowledges, i.e. non parametrical approach, the impulse response is defined by a sequence of values whose the number is related to the bounds of the frequency bandwidth. More specifically, in infrasonic system the lowest frequency is $0.02$ Hz, corresponding to periodicities around $50$ s. On the other hand the largest frequency leads to choose the sampling frequency to $20$ Hz. Therefore, in the absence of {\it a priori} knowledges, the impulse response consists of $50\times 20=1000$ real valued points. Hence the frequency response consists of $500$ complex valued points. However these values are obtained via statistical estimators and therefore several sequences  are required to reach a good accuracy.

There are basically two ways to estimate a linear filter either in time domain or in frequency domain. In both cases to get a good accuracy it is necessary to do averaging. The stochastic description is well-suited for that. The averaging can be applied using blocks of data or by adaptive approach (as Kalman filter or recursive least square). In the adaptive approach we adjust progressively the parameters of interest. This approach has been investigated leading to no interesting results. The main reason is the large variability of the observation. Indeed this kind of approach needs slow evolution of the stationarity. Numerical studies have shown that the best results are obtained if we take into account a small number of observation blocks and omit all the others.

In summary, in the following we only consider that the averaging is performed by block and we have adopted a frequency analysis. Basic tools for that is the spectral representation of wide sense stationary process (see annex \ref{ann:wss}).




%===========================================
\subsubsection{Remark on the deterministic approach (more details page \pageref{ann:deterministic-approach})}
Considering only time intervals with almost zero noises, we can ask: why do we use stochastic approach. Indeed the simple ratio of the short time Fourier transforms of the two observations, where the term short time is related to the long period, is the searched response. But there is also an equivalent to the MSC test. In the deterministic approach we have to validate the assumption that there is no noise and that in some way by locking if the performed ratio is almost constant along successive windows. That can be expressed by saying that the two sequences of Fourier along a few windows are correlated, and that leads to the MSC-like test.


%===========================================
\section{Observation model}
%===========================================
The notations are reported figure \ref{fig:general-schema}.  
%=================================
\subsection{Continuous time model}
%=================================

In the introduction discussion, we have explained that the model of signals are given by expressions \eqref{eq:model-signal-introduction}, which are re-written below:
\begin{eqnarray}
\label{eq:model-of-excitation}
\left\{
\renewcommand\arraystretch{1.6}
\begin{array}{rcl}
e_{\ut}(t)&=&s(t)+w_{\ut}(t)
\\
e_{\rf}(t)&=&s(t)+w_{\rf}(t)
\end{array}
\right.
\end{eqnarray}


The common signal $s(t)$ is the key of the estimation. On the other hand the noises induce a loss of identifiability and appear as a nuisance factor which can be characterized by a loss of coherence (LOC). \\*


It follows that the observation signals write:
\begin{eqnarray}
\label{eq:model-of-obervation}
\left\{
\renewcommand\arraystretch{1.6}
\begin{array}{rcl}
x_{\ut}(t)&=&g_{\ut} (t)\star(s(t)+w_{\ut}(t))
\\
x_{\rf}(t)&=&g_{\rf}(t) \star (s(t)+w_{\rf}(t))
\end{array}
\right.
\end{eqnarray}
%================ figure ====================
\figscale{schema-of-observation.pdf}{Basic observation schema. $\hat S_{UU}(f)$, $\hat S_{RR}(f)$ and $\hat S_{UR}(f)$ represent the three estimated spectral components under stationary assumptions}{fig:general-schema}{0.6}

The stationarity and the absence of correlation between $s(t)$, $w_{\ut}(t)$ and $w_{\rf}(t)$ play a fundamental  role in the identifiability of the response of the SUT. We assume that
\begin{itemize}
\item
$s(t)$ is a wide-sense stationary process with zero-mean and spectral density $\gamma_{s}(f)$,
which is the Fourier transform of the covariance fonction:
\begin{eqnarray*}
 R(\tau) = \esp{s(t+\tau)s(t)}&\rightleftharpoons  &
 S(f)=\int R(\tau) e^{-2j\pi f\tau}d\tau
\end{eqnarray*}
\item
$w_{u}(t)$ and $w_{r}(t)$ are two  stationary processes with zero-mean and respective spectral densities $\gamma_{\ut}(f)$ and $\gamma_{\rf}(f)$. 
\item
$s(t)$,  $w_{\ut}(t)$ and $w_{\rf}(t)$ are jointly independent. More specifically we have for any couple of time $(t,t')$:
\begin{eqnarray*}
\begin{array}{lccr}
 &\esp{w_{\ut}(t)w_{r}(t')}&=&0
 \\
& \esp{w_{\ut}(t)s(t')}&=&0
 \\
& \esp{w_{\rf}(t)s(t')}&=&0
\end{array}
 \end{eqnarray*}
\item
$G_{u}(f)$ represents the frequency response of the SUT which is the combination of the sensor under test, the frequency response of the anti-noise pipes and the digitalizer. $G_{u}(f)$ is the parameter to estimate.
\item
$G_{r}(f)$ represents the frequency response of the SREF. $G_{r}(f)$ is known and has been checked just before the installation.
\end{itemize}

Under these assumptions, the spectral content is fully described by the three components saying the  auto-spectrum of the signal of the SUT, denoted $S_{UU}(f)$, the auto-spectrum of the signal of the SREF, denoted  $S_{RR}(f)$ and the  cross-spectrum, denoted $S_{UR}(f)$. They are related to the input signals and the responses of the sensors by the three following expressions:
\begin{eqnarray}
\label{eq:spectral-model}
\left\{
\renewcommand\arraystretch{1.6}
\begin{array}{rcl}
S_{UU}(f)&=&|G_{\ut}(f)|^{2} (\gamma_{s}(f)+\gamma_{\ut}(f))
\\
S_{RR}(f)&=&|G_{\rf}(f)|^{2} (\gamma_{s}(f)+\gamma_{\rf}(f))
\\
S_{UR}(f)&=&G_{\ut}(f)G^{*}_{\rf}(f)\gamma_{s}(f)
\end{array}
\right.
\end{eqnarray}
It is worth to notice that the cross-spectrum does not depend on the noises because the assumptions of non correlation.


%=================================
\subsection{Discrete time model}
%=================================
We consider that the frequency band of interest is $(1/\tau_{c}, F_{M})$. $F_{M}$ allows to replace the continuous time signals by a time series with a sampling frequency $F_{s}=2F_{M}$. In infrasonic $F_{s}=20$ Hz. $\tau_{c}$ implies to estimate the response up to this period. In the infrasonic context $\tau_{c}=50$ seconds. That implies to reach in the  frequency domain the frequency  $1/\tau_{c}$. Taking a regular grid on the unit circle, as it is with FFT algorithm, the number $L$ of frequency bins will be of an order of magnitude of $\tau_{c}F_{s}$.That leads to take $L$ around $1000$. However in the following numerical studies we have investigated the response up to $500$ seconds, leading to FFT length of $10,\!000$.

 In the following the index $k$ refers to the frequency $kF_{s}/L$. For example $\gamma_{s,k}=\gamma_{s}(kF_{s}/L)$. Because the signal is real, all spectral representations have hermitian symmetry and it is only necessary to consider the positive part of the frequency band. Finally the spectrum at frequency 0 is real valued and will be omitted. Therefore the spectral representation is restricted to the values of the frequency index $k$ going from $1$ to $K=L/2$. \\*

From the expressions \eqref{eq:spectral-model} we obtain for the discrete Fourier transform of the spectral components:
\begin{eqnarray}
\label{eq:spectral-model-discrete}
\left\{
\renewcommand\arraystretch{1.6}
\begin{array}{rcl}
S_{UU,k}&=&|G_{\ut,k}|^{2} (\gamma_{s,k}+\gamma_{\ut,k})\geq 0
\\
S_{RR,k}&=&|G_{\rf,k}|^{2} (\gamma_{s,k}+\gamma_{\rf,k})\geq 0
\\
S_{UR,k}&=&G_{\ut,k}G^{*}_{\rf,k}\gamma_{s,k}\in\mathbb{C}
\end{array}
\right.
\end{eqnarray}
and from the expression \eqref{eq:MSC-continuous-frequency} the MSC expression in the discrete frequency domain:
\begin{eqnarray}
 \label{eq:exact-MSC}
\MSC_{k} = \frac{|{S_{UR,k}}|^{2}}{{S_{UU,k}}{S_{RR,k}}}
\end{eqnarray}

%Also for $\MSC_{k}$ very close to $1$ we can use any of the two formulas \eqref{eq:ratio-sup} or \eqref{eq:ratio-inf}, to estimate the frequency response in the discrete frequency domain. 







%===========================================
%===========================================
%===========================================
\subsection{Resolution of \eqref{eq:spectral-model-discrete} w.r.t. $G_{\ut,k}$}
%===========================================

It is worth to notice tat the last equation of the expression \eqref{eq:spectral-model-discrete} leads to the following expression of the phase of $G_{\ut,k}$:
\begin{eqnarray}
 \label{eq:phase-estimation}
 \arg G_{\ut,k}&=& \arg G_{\rf,k}- \arg S_{UR,k}
\end{eqnarray}

For the module of $G_{\ut,k}$, the problem is a little bit more complicate. Indeed the problem is under-determined in the sense where an infinity of solutions exists. 

\subsubsection{Case of known noise level ratio}
%=================================
In the particular case where we know the ratio between $\gamma_{u,k}$ and  $\gamma_{r,k}$, the equations \eqref{eq:spectral-model} can be solved w.r.t. to $G_{u,k}$. Denoting $\rho_{k}=\gamma_{r,k}/\gamma_{u,k}$ we easily derive:
\begin{eqnarray}
\label{eq:known-noise-ratio}
&&\hspace{-1cm}|G_{\ut,k}|=
\frac{1}{2}\,\sqrt{\frac{S_{UU,k}}{S_{RR,k}}}\,|G_{\rf, k}|\times 
\\&&\nonumber
\left(
(1-\rho_{k})\sqrt{\MSC_{k}}+\sqrt{4\rho_{k}+(1-\rho_{k})^{2}\MSC_{k}}
\right)
\end{eqnarray}
Typically $\rho_{k}$ could be considered as equal to the number of inlets in the noise reduction system, e.g. $96$.




%===========================================
\subsubsection{Under-determination}
In the equations \eqref{eq:spectral-model} the response $G_{\rf,k}$ is perfectly known, the unknows are $G_{\ut,k}\in\mathds{C}$, $\gamma_{s,k}\in\mathds{R}^{+}$,  $\gamma_{\ut,k}\in\mathds{R}^{+}$
$\gamma_{\rf,k}\in\mathds{R}^{+}$. 
 
It follows that, in absence of {\it a priori} knowledges,  the system \eqref{eq:spectral-model} is under-determined with $1$ degree of freedom (d.o.f.). That means that for all $k$ we can, for example,  choose arbitrarily the ratio $\rho_{k}=\gamma_{\ut,k}/\gamma_{\rf,k}$ from $0$ to infinity, then solve the systems \eqref{eq:spectral-model}. If one of the two noises is zero, i.e. the noise ratio is either 0 or infinite, the solution becomes also unique and is given by:





\begin{itemize}
\item
if $\gamma_{\ut,k}=0$, i.e. $\rho_{k}=0$
\begin{eqnarray}
\label{eq:ratio-sup-bis-exact}
{G_{\ut,k}}&=&R_{k}^{\sup}\,G_{\rf,k}, \quad \mathrm{with}\quad
R_{k}^{\sup} = \frac{{S_{UU,k}}}{{S^*_{UR,k}}}
\end{eqnarray}
\item
if $\gamma_{\rf,k}=0$, , i.e. $\rho_{k}=\infty$
\begin{eqnarray}
\label{eq:ratio-inf-bis-exact}
{G_{\ut,k}}&=&R_{k}^{\inf}\,G_{\rf,k}, \quad \mathrm{with}\quad
R_{k}^{\inf}=\frac{{S_{UR,k}}}{{S_{RR,k}}}
\end{eqnarray}
It is easy to show that $|R_{k}^{\inf}|\leq |R_{k}^{\sup}|$, using that the MSC is less than $1$.\\*
\end{itemize}

 Unfortunately there is no way to test if one of the two noises is zero. However it is possible to test that the two noises $w_{\ut,k}$ and $w_{\rf,k}$ are almost negligible. In this case the two estimators given by \eqref{eq:ratio-sup-bis-exact} and \eqref{eq:ratio-inf-bis-exact} are almost equal and almost unbiased. 

The assumptions of negligible noises can be tested by thresholding the Magnitude Square Coherence (MSC) defined by:
\begin{eqnarray}
 \label{eq:MSC-continuous-frequency}
 \MSC_{k} &=& \frac{|S_{UR,}|^{2}}{S_{UU,k}S_{RR,k}}
\end{eqnarray}
 
In the model \eqref{eq:spectral-model} MSC also writes:
\begin{eqnarray}
\label{eq:coherence-in-our-model}
 \MSC_{k} &=& \frac{1}{(1+\iSNR_{\ut,k})(1+\iSNR_{\rf,k})}
\end{eqnarray}
where $\iSNR_{\rf,k}=\gamma_{\rf,k}/\gamma_{s,k}$ and $\iSNR_{\ut,k}=\gamma_{\ut,k}/\gamma_{s,k}$. It follows that the MSC is 1, iff the two noises are zero. We have to keep in mind:
\begin{itemize}
\item
that these equations assume stationarity,
\item
that $G_{\ut,k}$ is performed during the period of time where the two noises are sufficiently low,
\item
that we have no access to the true values of the spectral components but only estimates, inducing dispersion. 
\end{itemize}



%===========================================
%===========================================
%===========================================
\section{Spectral analysis}
%===========================================
The classical moment method leads to replace in the  formulas \eqref{eq:exact-MSC}, \eqref{eq:phase-estimation}, \eqref{eq:ratio-sup-bis-exact}  and \eqref{eq:ratio-inf-bis-exact} the true spectral components by consistent estimates. In the absence of {\it a priori} knowledges that is given by a non parametrical spectral analysis. More details about the statistics of these quantities, as  the ratio of spectral components or the MSC, are presented in annexe \ref{ann:spectral-estimation}. 


%=================================
%===========================================

Based on that, we replace the spectral components $S_{UU,k}$, $S_{RR,k}$ and $S_{UR,k}$ by their consistent estimates $\widehat{S_{UU,k}}$, $\widehat{S_{RR,k}}$ and $\widehat{S_{UR,k}}$, obtained via a Welch's approach by averaging successive periodograms, with a typical overlap of $50\%$ and Hann's window. It is worth to notice that the number of frequency  bins is related to the long period signals whereas the accuracy of the estimates are related to the number of windows which are used for averaging the periodograms, typically we have chosen $5$ windows. 


Also we can find in the annex \ref{ann:spectral-estimation} the details to calculate the probability distributions of $\hMSC_{k}$ and of the ratios $\widehat{S_{UU,k}}/|\widehat{S_{UR,k}|}$ and $|\widehat{S_{UR,k}}|/\widehat{S_{RR,k}}$. We also provide Matlab programs that performs these distributions. 

%=================================
%=================================
\subsection{MSC test}
%=================================

It is worth noticing that we have only an estimate of the MSC not the true value. Therefore a test function has been determined to ensure with a given confidence level, typically $95\%$, to be over a target-value. The target-value is related to the accuracy we want on the estimation of the response of the SUT. In this case an important parameter is the  stationarity duration of the signals. For example we see on figure \ref{fig:allHest} that with $5$ windows, i.e. about 4 minutes (for long period of 50 seconds), the MSC must be over $0.97$ to ensure an RMSE of 5\% on the amplitude response. For such requirements the MSC threshold of the test is about $0.99$, see figure \ref{fig:MSCtestthreshold}.

It is worth to notice that the approach is strongly conservative. Indeed we keep only a few periods of signals, but that could be largely enough if we considered that the calibration has to be done once a year.


\figscale{allHest.pdf}{Simulation: for simulating the sensors are two different IIR(2,2) with important resonances are used. The commun coherent signal $s(t)$ is white. The length of the frequency analysis is $300$ seconds, i.e. $6000$ points. The spectral components are performed on $M$ times this duration. The root mean square error is performed by integrating on the full band. These curves have been obtained with the program {\tt estimHanalysis}.}{fig:allHest}{0.8}

\figscale{MSCtestthreshold.pdf}{Cumulative function of $\hMSC$ for different values of the number $M$ of averaging under the hypothesis that the MSC is $0.96$. These curves provide the threshold to test the hypothesis $H_{0}=\{\MSC>0.96\}$  with a confidence level of $90\%$.}{fig:MSCtestthreshold}{0.8}


  \newpage
 It is important to remark that an RMSE of 5\% means that the measurement has only a probability of $0.7$ to be in this interval, if we assume gaussian asymptotic behavior. Therefore we could expect that we consider at first an accuracy of $10\%$ and reduce in second step this number by aggregating many measurements. The problem is when the MSC is far from $1$ a large indetermination appears which does not leads to a gaussian asymptotic behavior \emph{around the true value}. That is reported on the theoretical curves of the figure \ref{fig:theoreticaldistribratios}. We see that the two ratios are distributed to a gaussian distribution but at a wrong location. It is not possible to correct this bias because that would assume that  the noise ratio is exactly know.


\figscale{theoreticaldistribratios.pdf}{Asymptotic distributions of the two ratios present in the formulas \eqref{eq:ratio-sup} and \eqref{eq:ratio-inf}. These distributions depend on the $\MSC$, but also on the noise levels and that is not known in practical case. These curves are obtained with the program {\tt CIHestimate.m}. The expressions are proved in the annex (see also the provided toolbox).}{fig:theoreticaldistribratios}{0.7}


%=================================
%=================================
\subsection{SUT estimation}
%=================================
In each time window with an MSC above the selected threshold, we can use any of the formulas \eqref{eq:ratio-sup-bis-exact} or  \eqref{eq:ratio-inf-bis-exact} replacing the true values by estimated values. The estimated values are distributed as it is reported figure \ref{fig:theoreticaldistribratios}.  The theoretical expression of the distributions of both ratios are given in annex \ref{ann:spectral-estimation}. In first approximation, we can assume for large values of MSC that the bias is zero and only stays the variance.

Therefore if we assume that the different values of the ratios along a large period of time are identically distributed and statistically independent, we can improve the accuracy by averaging. Also using the level of confidence of each ratio, i.e. its variance which is related to the  estimated MSC, we can perform a weighted average with weights in the inverse of this variance following:
\begin{eqnarray}
\label{eq:weithted-average-Ratio}
 \hat R &=& \frac{1}{L}\sum_{\ell=1}^{L}w_{\ell}\hat R_{\ell}
\end{eqnarray}
where $L$ denotes the number of periods of time with MSC above the selected threshold,
\begin{eqnarray}
\label{eq:estimated-Ratio}
\hat R_{\ell,k} ^{\sup}=\frac{\hat S_{UU,k}^{(\ell)}}{\hat S^{*(\ell)}_{UR,k}}
\quad
\mathrm{or}
\quad
\hat R_{\ell,k}^{\inf} =\frac{\hat S_{UR,k}^{(\ell)}}{\hat S_{RR,k}^{(\ell)}}
\end{eqnarray}
and where $w_{\ell}$ equals the inverse of the variances, whose approximate values are given by \eqref{eq:var12on22} and \eqref{eq:var11on21} and replacing true values by estimated values:
\begin{eqnarray}
\label{eq:weights}
\mathrm{for}\,\, R^{\sup}:  &&1/w_{\ell}=\frac{1}{2(2M+1)}
 \frac{\hat S_{UU,k}^{(\ell)}}{\hat S_{RR,k}^{(\ell)}} 
  \frac{1-\hat\MSC_{\ell}}{\hat\MSC_{\ell}^{2}}
\quad\mathrm{and}\\
\mathrm{for }\,\, R^{\inf}:  && 1/w_{\ell}= \frac{1}{2(2M+1)}
   \frac{\hat S_{UU,k}^{(\ell)}}{\hat S_{RR,k}^{(\ell)}} (1-\hat\MSC_{\ell})
\end{eqnarray}

That is implemented in the fonction {\tt fbankanalysis.m}, look at the flag {\tt weightingflag}.

\subsubsection{Remark}

It is worth to notice that this weighted average is a little bit optimist because based on the assumption that the distribution is identical and the estimates independent.
We might be tempted to use this assumption to identify the parameter of interest as it follows: we start from the equation \eqref{eq:spectral-model-discrete}, labeled by $\ell$ for different observed periods:
\begin{eqnarray}
\label{eq:spectral-model-discrete-ell}
\left\{
\renewcommand\arraystretch{1.6}
\begin{array}{rcl}
S_{UU,k}^{(\ell)}&=&|G_{\ut,k}|^{2} (\gamma_{s,k}+\gamma_{\ut,k})
\\
S_{RR,k}^{(\ell)}&=&|G_{\rf,k}|^{2} (\gamma_{s,k}+\gamma_{\rf,k})
\\
S_{UR,k}^{(\ell)}&=&G_{\ut,k}G^{*}_{\rf,k}\gamma_{s,k}
\end{array}
\right.
\end{eqnarray}
Under the assumption that the unlabeled unknown, saying $\gamma_{s,k}$, $\gamma_{\ut,k}$  and
$\gamma_{\rf,k}$ are identical, we can estimate these values. But it is likely that this approach is not fruitful. Indeed if we look at the distribution of  $R_{k}^{\sup}$ and $R_{k}^{\inf}$  as reported figure \ref{fig:practicalratiodistribution}, it is different from the theoretical distribution given figure \ref{fig:theoreticaldistribratios}. The  main reason is that we mixed many different values of the MSCs.

\figscale{practicalratiodistribution6.pdf}{Left: $R_{\inf}$, right: $R_{\sup}$. The values are obtained in the band of interest of the filter 3. The selected MSC threshold is $0.95$.}{fig:practicalratiodistribution}{0.8}


%================================
%================================
\section{Using a filter bank}
%================================
We had said that the spectral approach is based on stationarity property. But the real signals do not present permanent stationarity. Therefore we have to use time windows where this stationarity can be verified. Moreover is it maybe possible that in the high frequency range, saying e.g. around $1$ Hz, the time window length could be chosen shorter than in the low frequency range, saying e.g. around $0.02$ Hz. It is the main reason to propose a filter bank process in the full processing pipeline. The general pipeline proposed for the estimation process consists:
\begin{itemize}
\item
of a filter bank described in a file of settings which is characterized by a sequence of following descriptors: type, frequency lower bound, upper frequency bound, order, desired stationarity duration etc. Commonly used type is Butterworth (available on Matlab).
\item
of a spectral estimation process. At the output of a filter, the two signals (one for the SUT the other for the SREF) are shared in non-overlapping blocks to perform the spectral estimates. Longer the size of a block more accurate the spectral analysis. But that assumes stationarity, and the real signals do not exhibit permanent stationarity. Therefore we can choose in the setting file the desired stationarity duration in term of multiple of the longest period in the band (inverse of lower frequency bound). Typically we take  $5$ times the longest period, expecting that we can find such length with stationarity to estimate the spectral components. Even with that, the accuracy is not in accordance with the PTS requirements, but by averaging in a long period of times (many days), we can reach such requirements.
\item
Each block is then shared in overlapping sub-blocks. We can choose the windowing and the overlap rate. Typically we consider Hann windowing and $50\%$ of overlapping. The periodograms in the different sub-blocks are averaged to provide a spectral estimation. 

Only estimates in the bandwidth of the filter are retained.
\item
MSC is performed in each bandwidth. If the MSC is above the selected threshold, the value of the ratio $\hat S_{UU,k}/\hat S_{RU,k}$ is saved.
\item
along the recorded data, for a given frequency index $k$, the values  $\hat S_{UU,k}/\hat S_{RU,k}$ are averaged with weighting factors in accordance with $\hMSC$ values.
\item
Finally the estimation of the SUT response is derived from the SREF response.
\end{itemize}
The following shows an example of  process settings:
\begin{center}
\small
\begin{verbatim}
P=P+1;
filtercharact(P).designname = 'butter';
filtercharact(P).Norder=2;
filtercharact(P).Wlow_Hz=0.02;
filtercharact(P).Whigh_Hz=0.08;
filtercharact(P).SCPperiod_sec=100;
filtercharact(P).windowshape='hann';
filtercharact(P).overlapDFT=0.5;
filtercharact(P).overlapSCP=0;
filtercharact(P).ratioDFT2SCP=5;
\end{verbatim}
\end{center}
\begin{itemize}
\item
{\tt designname} means that the name of the filter model. The Butterworth model is available in Matlab. 
\item
{\tt Norder} denotes the order of the filter. If 0 there is no filtering. 
\item
{\tt Wlow\_Hz} denotes the low bound in Hz of the filter design,
\item
{\tt Whigh\_Hz} denotes the high bound in Hz of the filter design,
\item
{\tt windowshape} denotes the weighted window. Many windows are available in Matlab. Window is used to do a compromise between the leakage and the bias. Hann's window is commonly used.
\item
{\tt SCPperiod\_sec} denotes the time duration expressed in second of the window used for the spectral estimation,
\item
{\tt overlapDFT} overlapping rate for the spectral estimation,
\item
{\tt overlapSCP} overlapping rate for the different spectral estimates,
\item
{\tt ratioDFT2SCP} is an integer which denotes the ratio between the length of the duration fo spectral estimation and the duration for periodogram performing. 

For example if {\tt SCPperiod\_sec=500}, {\tt overlapDFT = 0.5} and  {\tt ratioDFT2SCP=5}, the FFTs are computed on $500/5=100$ seconds, then for a sampling frequency of 20 Hz, on 2000 samples. Because the overlap is 0.5 we compute 9 FFTs.
\end{itemize}

\figscale{overlapFFTs.pdf}{Spectral estimation with $50\%$ overlapping. The variance of the estimate is related to the number of windows. The frequency resolution is related to the length of the FFT which is $100$ seconds, hence $0.01$ Hz. The window shape is related to the leakage which is defined as the effect of transferring energy from the bands where the energy is high to the bands where the energy is low. }{}{0.8}
%=================================
\subsubsection{Filter bank analysis}
%=================================
Let us recall that in a first step, the signals are filtered in adjacent bands in such a way to use different periods whose the main interest is to consider short periods, if necessary, in the high frequency bands. The table \ref{tab:freq-duration-tradeoff} is an example of pavement, consisting of $5$ bands with log-spaced filter parameters in the band $(0.01-5)$ Hz with a variable window length\footnote{See also recent PMCC reports.}. Because the two signals are applied to the same filter we can use RII as Butterworth filter. Also because this process is for anlysis, we don't need to downsampling and/or to require perfect recontruction. Even more the bandpass filters can be overlap in the frequency domain. Although a decimation can be used to save computational time, indeed the bandwidths are lower than $F_{s}/2$, that operation is not considering in the following.

\begin{table}
\begin{center}
\begin{tabular}{|c|c|}
\hline
frequency band (Hz) & stationarity period (second)
\\
\hline
%%%%% from matlab
$[0.02-0.20]$&$400$
\\ \hline $[0.20-1.00]$&$200$
\\ \hline $[1.00-2.00]$&$100$
\\ \hline $[2.00-3.00]$&$50$
\\ \hline $[3.00-4.00]$&$25$
\\ \hline $[4.00-6.00]$&$25$
\\ \hline 
%%%%
\end{tabular}
\parbox{12 cm}
{
    \caption{\protect\small\it  }
    \label{tab:freq-duration-tradeoff}
}
\end{center}
\end{table}
\figscale{filterbank.pdf}{Filter bank}{fig:filterbank}{1}


%================================

