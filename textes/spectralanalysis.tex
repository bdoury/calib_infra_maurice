% !TEX root = ../distCjHMSC.tex
%===========================================
%===========================================
\section{Observation model}
%===========================================
The notations are reported figure \ref{fig:general-schema}. The input signals write:
\begin{eqnarray}
\label{eq:model-of-excitation}
\left\{
\renewcommand\arraystretch{1.6}
\begin{array}{rcl}
e_{\ut}(t)&=&h(t) \star s(t)+w_{\ut}(t)
\\
e_{\rf}(t)&=&s(t)+w_{\rf}(t)
\end{array}
\right.
\end{eqnarray}
where $h(t)$ would take into account some possible filtering effect. It is worth to notice that because $s(t)$ is unknown, the unknown filter $h(t)$ is not identifiable, even in the simple case where $h(t) \star s(t)=\alpha s(t)$. In the following we only consider the case where $h(t)=\delta(t)$, leading to:
\begin{eqnarray}
\label{eq:model-of-excitation}
\left\{
\renewcommand\arraystretch{1.6}
\begin{array}{rcl}
e_{\ut}(t)&=&s(t)+w_{\ut}(t)
\\
e_{\rf}(t)&=&s(t)+w_{\rf}(t)
\end{array}
\right.
\end{eqnarray}


The common signal $s(t)$ can be seen as the coherent part of the input signals. It is the key of the estimation. On the other hand the noises induce a loss of identifiability and appear as a nuisance factor which can be characterized by a loss of coherence (LOC). It follows that the observation signals write:
\begin{eqnarray}
\label{eq:model-of-obervation}
\left\{
\renewcommand\arraystretch{1.6}
\begin{array}{rcl}
x_{\ut}(t)&=&g_{\ut} \star(s(t)+w_{\ut}(t))
\\
x_{\rf}(t)&=&g_{\rf} \star (s(t)+w_{\rf}(t))
\end{array}
\right.
\end{eqnarray}
%================ figure ====================
\figscale{schema-of-observation.pdf}{Basic observation schema. $\hat S_{UU}(f)$, $\hat S_{RR}(f)$ and $\hat S_{UR}(f)$ represent the three estimated spectral components under stationary assumptions}{fig:general-schema}{0.6}

The stationarity and the absence of correlation between $s(t)$, $w_{\ut}(t)$ and $w_{\rf}(t)$ play a fundamental  role in the identifiability of the response of the SUT. We assume that
\begin{itemize}
\item
$s(t)$ is a wide-sense stationary process with zero-mean and spectral density $\gamma_{s}(f)$,
\item
$w_{u}(t)$ and $w_{r}(t)$ are two  stationary processes with zero-mean and respective spectral densities $\gamma_{\ut}(f)$ and $\gamma_{\rf}(f)$. 
\item
$s(t)$,  $w_{\ut}(t)$ and $w_{\rf}(t)$ are jointly independent. More specifically we have for any $(t,t')$:
$$
 \esp{w_{\ut}(t)w_{r}(t')}=0
$$
\item
$G_{u}(f)$ represents the frequency response of the SUT which is the combination of the sensor under test, the frequency response of the anti-noise pipes and the digitalizer. It is the parameter to estimate.
\item
$G_{r}(f)$ represents the frequency response of the SREF. $H_{r}(f)$ is known and could be checked just before the test operation.
\end{itemize}



%===========================================
%===========================================
%===========================================
\section{SUT estimation}
%===========================================
If we assume that the SUT is well modeled by a linear filter, a common way to calibrate it is to estimate its impulse response or equivalently its frequency response. In the absence of {\it a priori} knowledges, i.e. non parametrical approac, the only parameters are the bounds of the frequency bandwidth. The lowest value is related to the largest period we want to analyze and the highest leads to the sampling frequency. In infrasonic system the largest period is about 50 seconds. Therefore to expect a good accuracy we can assume that at least 5 times this value is necessary leading to about 4 minutes. In frequency domain that is equivalent to about $0.004$ Hz. In infrasonic system the largest frequency leads to choose the sampling frequency to $20$ Hz. In summary impulse response consists of $4800$ real values and frequency response consists of $2400$ complex values. 


There are basically two ways to estimate a linear filter either in time domain or in frequency domain. In both cases to get a good accuracy it is necessary to do averaging. The stochastic description is well-suited for that. The averaging can be applied using blocks of data or by adaptive approach (as Kalman filter or recursive least square). In the adaptive approach we adjust progressively the parameters of interest. This approach has been investigated leading to no interesting results. The main reason is the large variability of the observation. Indeed this kind of approach needs slow evolution of the stationarity. Numerical studies have shown that the best results are obtained if we take into account a small number of observation blocks and omit all the others.

In summary, in the following we only consider that the averaging is performed by block and we have adopted a frequency analysis. Basic tools for that is the spectral representation of wide sense stationary process (see annex \ref{ann:wss}).


If we adopt a stochastic approach and assume that the processes $s(t)$, $w_{\ut}(t)$ and $w_{\rf}(t)$ are jointly stationary and jointly uncorrelated, and the two sensors are time-invariant, the spectral content is described by the following spectral equations:
\begin{eqnarray}
\label{eq:spectral-model}
\left\{
\renewcommand\arraystretch{1.6}
\begin{array}{rcl}
S_{UU}(f)&=&|G_{\ut}(f)|^{2} (\gamma_{ss}(f)+\gamma_{\ut}(f))
\\
S_{RR}(f)&=&|G_{\rf}(f)|^{2} (\gamma_{ss}(f)+\gamma_{\rf}(f))
\\
S_{UR}(f)&=&G_{\ut}(f)G^{*}_{\rf}(f)\gamma_{ss}(f)
\end{array}
\right.
\end{eqnarray}
$S_{UU}(f)$, $S_{RR}(f)$ and $S_{UR}(f)$ are respectively the auto-spectrum of the signal of the SUT, the auto-spectrum of the signal of the SREF and the cross-spectrum between them. It is worth to notice that the cross-spectrum does not depend on the noises because the assumption of non correlation.

%===========================================
\subsubsection{Under-determination}
In the equations \eqref{eq:spectral-model} the response $G_{\rf}(f)$ is perfectly known, the unknows are $G_{\ut}(f)\in\mathds{C}$, $\gamma_{ss}(f)\in\mathds{R}^{+}$,  $\gamma_{\ut}(f)\in\mathds{R}^{+}$
$\gamma_{\rf}(f)\in\mathds{R}^{+}$. 
 
It follows that, in absence of {\it a priori} knowledges,  the system \eqref{eq:spectral-model} is under-determined with $1$ degree of freedom (d.o.f.). That means that for all $f$ we can, for example,  choose arbitrarily the ratio $\rho(f)=\gamma_{\ut}(f)/\gamma_{\rf}(f)$ from $0$ to infinity, then solve the systems \eqref{eq:spectral-model}. If one of the two noises is zero, i.e. the noise ratio is either 0 or infinite, the solution becomes unique and is given by:
\begin{itemize}
\item
if $\gamma_{\ut}(f)=0$, i.e. $\rho(f)=0$
\begin{eqnarray}
\label{eq:ratio-sup}
G_{\ut}(f)&=&
\frac{S_{UU}(f)}{S_{UR}^*(f)}\,G_{\rf}(f)
\end{eqnarray}
\item
if $\gamma_{\rf}(f)=0$, , i.e. $\rho(f)=\infty$
\begin{eqnarray}
\label{eq:ratio-inf}
G_{\ut}(f)&=&
\frac{S_{UR}(f)}{S_{RR}(f)}\,G_{\rf}(f)
\end{eqnarray}
\end{itemize}
 Unfortunately there is no way to test if one of the two noises is zero.


To circumvent this drawback we consider a more constrained condition: both noises $w_{\ut}(t)$ and $w_{\rf}(t)$ are almost negligible. In this case the two estimators given by \eqref{eq:ratio-sup} and \eqref{eq:ratio-inf} are almost equal and almost unbiased. 

The assumptions of negligible noises can be tested by thresholding the Magnitude Square Coherence (MSC) defined by:
\begin{eqnarray}
 \label{eq:MSC-continuous-frequency}
 \MSC(f) &=& \frac{|S_{UR}(f)|^{2}}{S_{UU}(f)S_{RR}(f)}
\end{eqnarray}
 
It can be shown that the MSC is 1, iff the two noises are zero.\\*

In the model \eqref{eq:spectral-model} MSC also writes:
\begin{eqnarray}
\label{eq:coherence-in-our-model}
 \MSC(f) &=& \frac{1}{(1+\iSNR_{\ut}(f))(1+\iSNR_{\rf}(f))}
\end{eqnarray}
where $\iSNR_{\rf}(f)=\gamma_{\rf}(f)/\gamma_{ss}(f)$ and $\iSNR_{\ut}(f)=\gamma_{\ut}(f)/\gamma_{ss}(f)$.

%===========================================
\subsubsection{Remark on the deterministic approach (a few details page \pageref{ann:deterministic-approach})}
Considering only time intervals with almost zero noises, we can ask: why do we use stochastic approach. Indeed the simple ratio of the short time Fourier transforms of the two observations, where the term short time is related to the long period, is the searched response. But there is also an equivalent to the MSC test. In the deterministic approach we have to validate the assumption that there is no noise and that in some way by locking if the performed ratio is almost constant along successive windows. That can be expressed by saying that the two sequences of Fourier along a few windows are correlated, and that leads to the MSC-like test.

%===========================================
%===========================================
%===========================================
\section{Spectral analysis}
%===========================================
The classical moment method leads to replace in the  formulas like \eqref{eq:ratio-sup}  or \eqref{eq:ratio-inf} the true spectral components by consistent estimates. In the absence of {\it a priori} knowledges that is given by a non parametrical spectral analysis. More details about the statistics of these quantities, as  the ratio of spectral components or the MSC, are presented in annexe \ref{ann:spectral-estimation}. 


%===========================================
\subsubsection{Discrete time model}
We consider that the frequency band of interest is $(1/\tau_{c}, F_{M})$. $F_{M}$ allows to replace the continuous time signals by a time series with a sampling frequency $F_{s}=2F_{M}$. In infrasonic $F_{s}=20$ Hz. $\tau_{c}$ implies to estimate the response up to this period. In the infrasonic context $\tau_{c}=50$ seconds. That implies to reach in the normalized frequency domain the frequency of about $1/\tau_{c}F_{s}=1/1000$. Taking a regular grid on the unit circle, as it is with FFT algorithm, the number $L$ of frequency bins will be of an order of magnitude of a few times $\tau_{c}F_{s}$. If we consider an order of 4 to 5, that leads to take $L$ around $4096$.

 In the following the index $k$ refers to the frequency $kF_{s}/L$. For example $\gamma_{ss,k}=\gamma_{ss}(kF_{s}/L)$. Because the signal is real, all spectral representations have hermitian symmetry and it is only necessary to consider normalized frequency between $0$ and $1/2$. Finally the spectrum at frequency 0 is real valued and will be omitted. Therefore the spectral representation is restricted to the values of the frequency index $k$ going from $1$ to $K=L/2$. 

Let us rewrite the spectral components
\begin{eqnarray}
\label{eq:spectral-model-dicrete}
\left\{
\renewcommand\arraystretch{1.6}
\begin{array}{rcl}
S_{UU,k}&=&|G_{\ut,k}|^{2} (\gamma_{ss,k}+\gamma_{\ut,k})\geq 0
\\
S_{RR,k}&=&|G_{\rf,k}|^{2} (\gamma_{ss,k}+\gamma_{\rf,k})\geq 0
\\
S_{UR,k}&=&G_{\ut,k}G^{*}_{\rf,k}\gamma_{ss,k}\in\mathbb{C}
\end{array}
\right.
\end{eqnarray}
From \eqref{eq:MSC-continuous-frequency} we derive the MSC expression in the discrete time case:
\begin{eqnarray}
 \label{eq:exact-MSC}
\MSC_{k} = \frac{|{S_{UR,k}}|^{2}}{{S_{UU,k}}{S_{RR,k}}}
\end{eqnarray}
and, for $\MSC_{k}$ very close to $1$ we can use any of the two formulas \eqref{eq:ratio-sup} or \eqref{eq:ratio-inf}, giving:
\begin{eqnarray}
\label{eq:ratio-sup-bis-exact}
{G_{\ut,k}}&=&
\frac{{S_{UU,k}}}{{S^*_{UR,k}}}\,G_{\rf,k}
\end{eqnarray}
or
\begin{eqnarray}
\label{eq:ratio-inf-bis-exact}
{G_{\ut,k}}&=&
\frac{{S_{UR,k}}}{{S_{RR,k}}}\,G_{\rf,k}
\end{eqnarray}

We have to keep in mind:
\begin{itemize}
\item
that these equations assume stationarity,
\item
that we perform the value of $G_{\ut,k}$ during the period where the 2 noises are sufficiently low,
\item
that we have no access to the true values of the spectral components but only estimates, inducing dispersion. 
\end{itemize}

Based on that, we can replace in expressions \eqref{eq:exact-MSC} and \eqref{eq:ratio-sup-bis-exact} the spectral components $S_{UU,k}$, $S_{RR,k}$ and $S_{UR,k}$ by their consistent estimates $\widehat{S_{UU,k}}$, $\widehat{S_{RR,k}}$ and $\widehat{S_{UR,k}}$.


These consistent estimates are obtained via a Welch's approach by averaging successive periodograms, with a typical overlap of $50\%$ and Hann's window. It is worth to notice that the number of frequency  bins is related to the long period signals whereas the accuracy of the estimates are related to the number of windows which are used for averaging the periodograms.


Also we can find in the annex \ref{ann:spectral-estimation} the details to calculate the probability distributions of $\hMSC$ and of the ratio $\widehat{S_{UU,k}}/\widehat{S^*_{UR,k}}$. We also provide Matlab programs that performs these distributions. 

%=================================
\subsubsection{MSC approach}
%=================================

It is worth noticing that we have only an estimate of the MSC not the true value. Therefore a test function has been determined to ensure with a given confidence level, typically $95\%$, to be over a target-value. The target-value is related to the accuracy we want on the estimation of the response of the SUT. In this case an important parameter is the  stationarity duration of the signals. For example we see on figure \ref{fig:allHest} that with $5$ windows, i.e. about 4 minutes (for long period of 50 seconds), the MSC must be over $0.97$ to ensure an RMSE of 5\% on the amplitude response. For such requirements the MSC threshold of the test is about $0.99$, see figure \ref{fig:MSCtestthreshold}.

It is worth to notice that the approach is strongly conservative. Indeed we keep only a few periods of signals, but that could be largely enough if we considered that the calibration has to be done once a year.


\figscale{allHest.pdf}{Simulation: for simulating the sensors are two different IIR(2,2) with important resonances are used. The commun coherent signal $s(t)$ is white. The length of the frequency analysis is $300$ seconds, i.e. $6000$ points. The spectral components are performed on $M$ times this duration. The root mean square error is performed by integrating on the full band. These curves have been obtained with the program {\tt estimHanalysis}.}{fig:allHest}{1}

\figscale{MSCtestthreshold.pdf}{Cumulative function of $\hMSC$ for different values of the number $M$ of averaging under the hypothesis that the MSC is $0.96$. These curves provide the threshold to test the hypothesis $H_{0}=\{\MSC>0.96\}$  with a confidence level of $90\%$.}{fig:MSCtestthreshold}{1}


  \newpage
 It is important to remark that an RMSE of 5\% means that the measurement has only a probability of $0.7$ to be in this interval, if we assume gaussian asymptotic behavior. Therefore we could expect that we consider at first an accuracy of $10\%$ and reduce in second step this number by aggregating many measurements. The problem is when the MSC is far from $1$ a large indetermination appears which does not leads to a gaussian asymptotic behavior \emph{around the true value}. That is reported on the theoretical curves of the figure \ref{fig:theoreticaldistribratios}. We see that the two ratios are distributed to a gaussian distribution but at a wrong location. It is not possible to correct this bias because that would assume that  the noise ratio is exactly know.


\figscale{theoreticaldistribratios.pdf}{Asymptotic distributions of the two ratios present in the formulas \eqref{eq:ratio-sup} and \eqref{eq:ratio-inf}. These distributions depend on the $\MSC$, but also on the noise levels and that is not known in practical case. These curves are obtained with the program {\tt CIHestimate.m}. The expressions are proved in the annex (see also the provided toolbox).}{fig:theoreticaldistribratios}{1}


%=================================
\subsubsection{SUT estimation}
%=================================
In each time window with an MSC above the selected threshold, we can use any of the formulas \eqref{eq:ratio-sup-bis-exact} or  \eqref{eq:ratio-inf-bis-exact} replacing the true values by estimated values. The estimated values are distributed as it is reported figure \ref{fig:theoreticaldistribratios}.  The theoretical expression of the distributions of both ratios are given in annex \ref{ann:spectral-estimation}. In first approximation, we can assume for large values of MSC that the bias is zero and only stays the variance.

Therefore if we assume that the different values along a large period of time are statistically independent, we can access to a more accurate value of the useful ratio by averaging. But because we select time segments with different MSCs, we can use the variance of the estimated ratio and compute an averaging with weights in the inverse of this variance following:
\begin{eqnarray}
\label{eq:weithted-average-Ratio}
 \hat R &=& \frac{1}{L}\sum_{\ell=1}^{L}w_{\ell}\hat R_{\ell}
\end{eqnarray}
where $L$ denotes the number of periods of time with MSC above the selected threshold,
\begin{eqnarray}
\label{eq:estimated-Ratio}
\hat R_{\ell} ^{\sup}=\frac{\hat S_{UU,k}}{S^{*}_{UR,k}}
\quad
\mathrm{or}
\quad
\hat R_{\ell}^{\inf} =\frac{\hat S_{UR,k}}{S_{RR,k}}
\end{eqnarray}
and where $w_{\ell}$ equals the inverse of the variances, whose approximate values are given by \eqref{eq:var12on22} and \eqref{eq:var11on21} and replacing true values by estimated values:
\begin{eqnarray}
\label{eq:weights}
\mathrm{for}\,\, R^{\sup}:  &&1/w_{\ell}=\frac{1}{2(2M+1)}\frac{\hat\Gamma_{1,1}}{\hat\Gamma_{2,2}}
  \frac{1-\hat\MSC}{\hat\MSC^{2}}
\quad\mathrm{or}\\
\mathrm{for }\,\, R^{\inf}:  && 1/w_{\ell}= \frac{1}{2(2M+1)}
   \frac{\hat\Gamma_{1,1}}{\hat\Gamma_{2,2}} (1-\hat\MSC)
\end{eqnarray}

That is implemented in the fonction {\tt fbankanalysis.m}, look at the flag {\tt weightingflag}.



