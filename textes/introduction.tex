% !TEX root = ../distCjHMSC.tex
%===========================================
%===========================================
\section{Objective}
%===========================================
\change[Benoit]{This study is devoted to the calibration of a sensor (assumed to be a linear time-invariant device). 
We mainly focus on the infrasonic stations but most results can extended to seismic stations.\\*}
{This study is devoted to the calibration of an infrasound system (Sensor + Wind Noise Reduction System),assumed to be a linear time-invariant system.
 It is expected that most results could be applied to the calibration of seismometers, using a reference instrument.\\*}

\change[Benoit]
{The PTS specifications for infrasonic frequency response sensor are $\pm 1\%$ on the amplitude gain and $\pm 5\degree$ on the phase in the frequency bandwidth going from $0.02$ Hz up to $4$ Hz}
{The Minimum Requirements for calibration set out in the Operational Manual \footnote{CTBT/WGB/TL-11,17/17/REV.5 Operational Manual for Infrasound Monitoring and the International Exchange of Infrasound Data} sets specifications for the frequency response to be within $\pm 5\%$ in amplitude gain from the nominal response, in the PTS frequency band of interest going from $0.02$ Hz up to $4$ Hz}. That represents a challenging issue.\\*

\change[Momo]{
The method to estimate the response of a sensor, denoted by sensor under test (SUT), is based on the use of a second sensor, denoted reference sensor (SREF), whose the frequency response is perfectly}{tutu} The method is based on the presence of a quasi permanent background acoustical sound all over the Word. The term "acoustical'' refers to the propagation velocity of this wave which is commonly around 340 m/s inducing in the whole frequency domain of interest large wavelengths compared to the size of the sensor. It follows that this signal appears coherently on the two sensors. In the following it is shortly called the coherent signal and is denoted $s(t)$. It has been observed and seems have a large enough frequency bandwidth to satisfy the PTS requirements. A general schema is reported figure \ref{fig:general-schema}. 
Unfortunately this signal is not observable. It follows that, in presence of additive noises, the problem is ill-conditioned. That means that they are an infinity of possible solutions.

A special mention concerns the noises denoted $w_{\ut}$ and $w_{\rf}$ (see figure \ref{fig:general-schema}). They are mainly due to the wind. Although the wind structure is very different of the  "acoustical'' signal $s(t)$, we will see that a parameter of interest is $\zeta=v/f$ where $v$ is the wind velocity and $f$ the frequency. $\zeta$ can be interpreted 
as a wavelength. For typical values of the wind velocity, $\zeta$ is much lower than the distances between air inlets of the noise reduction system (NRS).

%===========================================
\section{Experimental testbed}
%===========================================
An experimental testbed has been deployed in the IS26 located in Freyung in Germany:
\begin{itemize}
\item
The sensors under test/reference are reported table \ref{tab:sensor-specifications}. The reference sensors are deployed at a few meters of the sensors under test. These sensors have been checked before installation. We have two kinds of SREF, one is MB32005 with a short period, the other is MB3 with long period. The SUT are all MB3 with long period.
\item
The environment is particularly  quiet, due to the Bavarian Forest where the station is located. It follows that the wind is almost absent during large periods of time.

 \item
The data are recorded in continuous.
\end{itemize}

It is worth to notice that the calibration problem is a little bit  different of the detection problem which consists to provide an alert when the system is out of specifications.


\begin{table}
\begin{center}
\begin{tabular}{|c||cc|cc||}
\hline
Site & SUT & info.& SREF & info.
\\
\hline
     & model&serie \#     & model&serie \#
\\
H1/C1& MB3 & 00020  & MB2005& 6046
\\
H2/C2& MB3  &  00017  & MB2005& 5125
\\
H3/C3& MB3  &  00014  & MB2005 &6039
\\
H4/C4& MB3  &  00023  & MB2005& 5104
\\
H5/C5& MB3   & 00012  & MB2005 &7095
\\
H6/C6& MB3  &  00011  & MB3 &00007
\\
H7/C7& MB3   & 00021  & MB3 &00008
\\
H8/C8& MB3   & 00015  & MB3 &00022
\\
\hline
\end{tabular}
\parbox{12 cm}
{
    \caption{\protect\small\it  sensor specifications: all sensors have the same theoretical sensitivity of $0.02$ V/Pa. The MB3s have self noise lower than this of the MB2005s. The sites associated to the ''Wind Noise Reduction System" are labelled H whereas  the sites od reference sensors are labelled C.}
    \label{tab:sensor-specifications}
}
\end{center}
\end{table}

%\figscale{H1-H5MB2005.jpg}{}{}{fig:H1-H5MB2005}

 \newpage
%===========================================
%===========================================
\section{Main issue}
%===========================================
The main issue is related to the under-determination of the problem (also known as blind identification). More specifically we have four unknowns for only two observations. The four unknowns are the frequency response of the SUT, the spectral shapes of the coherent signal and the two noises on the two sensors. The two observations are the output signals on both sensors.

A common way to solve this kind of problem is to introduce some {\it a priori} knowledges/assumptions. The drawback is what happens when the {\it a priori} knowledges are not well-verified.

Here a list of {\it a priori} knowledges
\begin{itemize}
\item
 the signals are stationary in the whole frequency bandwidth of interest. More specifically we have to precise what we mean in term of stationarity duration. The worst case  is for the low frequency, i.e. for long period. For example if the accuracy requires to integrate the signal over five periods and if we want to analyze the spectral content around the frequency of $0.01$ Hz, about $10$ minutes of stationarity are needed.
 
Let us notice that, even under the stationarity assumption the problem is still under-determined.
 

  \item
the coherent signal and the noises can be modeled with parametrical models such as auto-regressive paradigm. In \cite{frazier:2013} a generalized autoregressive conditionally heteroskedastic (GARCH) model has been used to model the wind. This approach has not been investigated here.

  \item
 the noises on the two sensors are uncorrelated and white. This is the simplest parametrical model which depends on only one parameter. In this case the frequency response is identifiable, but observations and also the physical aspects of the wind noises deny this assumption.
   \item
the system under test is a linear filter with {\it given} numbers of poles and zeroes (parametrical approach). This approach has not been already investigated. It follows that, for the calibration, imply the numbers of poles and zeroes could hide a singularity and then induce a bias. To illustrate saying that a system is of order 1 can hide in the estimate the presence of a non suspected resonance. But in the alert issue framework, we  see that something is wrong, therefore we suspect that it could be an efficient way for test. 

\item
assuming that the two noises are uncorrelated, the indetermination disappears if  we assume that the ratio between the two noise levels is known. That is a realistic way in high frequency because the NRS works well and the ratio is about the inverse of the number of inlets.

\item
assuming that the two noises are uncorrelated, the indetermination disappears if  we assume that one of the two noises is negligible, typically it could be the case for the SUT which is provided with a noise reduction system (NRS). But we also want to detect if this NRS is working well.

The advantage of the  assumption is the second noise could be very large, we have only to use the formula \eqref{eq:ratio-sup}. The drawback is that no test does exist to provide an answer to the question: is the noise on the SUT negligible? Therefore if the assumption becomes wrong the error could be very large.

\item
assuming that the two noises are uncorrelated, the indetermination can be a fortiori removed if both noises are negligible. The advantage of this second assumption is double: (i) we can use any of the two formulas \eqref{eq:ratio-sup} or \eqref{eq:ratio-inf}, and (ii) we have an efficient way to test if or not the two noises are negligible. The main drawback is it could be difficult to find time windows where the condition is verified for a longperiod, particularly in the high frequency band. A solution is to consider shorter window sizes in the high frequency ranges. 



\end{itemize}

In summary, our conclusion is that, to get the expected accuracy, we have to consider that both noises are negligible. To test that the MSC (see below for specific definition) is used along time windows whose lengths vary in relation with the frequency ranges. From picturing manner, the MSC measures the fact that the estimated values exhibit very low dispersion, hence low noises, along a few consecutive windows. 
 
