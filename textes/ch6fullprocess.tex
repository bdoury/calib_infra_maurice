% !TEX root = ../calibreport.tex


%================================
\section{3 programs for test}
The full process consists of
\begin{enumerate}
\item
 move data from the IDC, using {\tt RUNextractfromDB.m}. The data with Matlab format is saved. This data consists of a structure called {\tt records} with the following items:
 \begin{itemize}
 \item
{\tt data}: sequence of samples,
 \item
{\tt Fs\_Hz}: sampling frequency,
\item
{\tt stime}, {\tt time}: start and end times, 
\item
{\tt station}: I26C1--I26C8 or I26H1--I26H8
\item
{\tt channel}: 'BDF' or 'LKO' or 'LWD' or 'LWS'
 \end{itemize}   
Typically in the following we use BDF on H and C.   The three other channels are only provided on H1. Channel LWD yields  the wind speed measurement. The wind sensor is located at about 2 meters above the infrasonic sensor. The frequency sampling depends on the channel. Typically for infrasound signals, the frequency sampling is $20$ Hz, and for wind features is 1 Hz.
    
\item
 using the filtercharacteristics, saved in {\tt filtercharacteristics.mat} and the directory which contains the data from one of the 8 channels, run the program {\tt estimationwithFB.m}. This program saves the useful elements for a display in a selected directory.
 \item
 to display run the program {\tt displaySUTresponse.m} for the file recorded in the selected directory of the previous item.
\end{enumerate}


%================================
\section{Input of the function {\tt fbankanalysis.m}}
\begin{itemize}
\item
signals: an array of size $N\times 2$ where $N$ is a number of samples, the first column is the signal from the SUT channel and the second for the SREF channel,
\item
filter bank settings: see section \ref{sss:filter-bank-settings},
\item
$F_s$: sampling frequency in Hz, typically $20$ Hz,
\item
MSC threshold: threshold for the MSC, typically $0.98$
\end{itemize}
\subsubsection{Summary of the filter bank settings}

\label{sss:filter-bank-settings}

\begin{center}
{\small\verbatiminput{filtercharacteristicsEXAMPLE.m}}
\end{center}

\begin{itemize}
\item
{\tt designname} means that the name of the filter model. The Butterworth model is available in Matlab. 
\item
{\tt Norder} denotes the order of the filter. If 0 there is no filtering. 
\item
{\tt Wlow\_Hz} denotes the low bound in Hz of the filter design,
\item
{\tt Whigh\_Hz} denotes the high bound in Hz of the filter design,
\item
{\tt windowshape} denotes the weighted window. Many windows are available in Matlab. Window is used to do a compromise between the leakage and the bias. Hann's window is commonly used.
\item
{\tt SCPperiod\_sec} denotes the time duration expressed in second of the window used for the spectral estimation,
\item
{\tt overlapDFT} overlapping rate for the spectral estimation,
\item
{\tt overlapSCP} overlapping rate for the different spectral estimates,
\item
{\tt ratioDFT2SCP} is an integer which denotes the ratio between the duration of the spectral estimation window and the duration of the DFT window. 

\end{itemize}


%================================
\section{Output of the function {\tt fbankanalysis.m}}

\begin{verbatim}
% 1) SUTs: structures P x 1
%         xx.estimRsup: Rsup ratio
%         xx.estimRinf: Rinf ratio
%         xx.allMSCs: allMSCs;
%         xx.Nsupthreshold: count of the number of values over
%                   the threshold
%         xx.Nsupthresholdintheband: counts of the number of values
%                   over the threshold in the filter bandwidth
%         xx.frqsFFT_Hz: cell P x 1, each cell consists of
%                   frequency list in Hz of the DFTs
%         xx.SCP = all spectral components
%         xx.indexinsidefreqband = P x 1, indices of the
%                    frequency bounds of each filter
%                    in the xx.frqsFFT_Hz
% 2) filteredsignals, 
% 3) allfrqsFFT_Hz, cell P x 1, each cell for each bank filter consists of
%             frequency list in Hz of the DFTs
% 4) alltimes_sec: cell  Px 1, each cell consists of the structure
%             yy.FFT:     time list in second of the DFTs
%             yy.SD:      time list in second of the SCPs
%             yy.signals: time list in second of the signals
% 5) filterbank



\end{verbatim}



