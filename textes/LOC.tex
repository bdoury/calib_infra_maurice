% !TEX root = ../distCjHMSC.tex
%============================================

Let us consider the equation:
$$
 Y_{t}=g_{t} \star X_{t}
$$
where $g_{t}$ is the $M$-ary impulse response of the single input multiple output (SIMO) linear filter and $X_{t}$ is a scalar WSS process, with zero mean and spectral matrix (of size 1) $\gamma_{x}(f)$. It is easy to show that the spectral matrix of $Y_{t}$ writes
$$
 \Sigma_{y}(f)=\gamma_{x}(f)G(f)G^{H}(f)
$$
where $G(f)$ is the discrete time Fourier transform of $g_{t}$.
We notice that $\Sigma_{y}(f)$ is a matrix of rank 1. A particular case is a pure delay propagation filter as for example in the planar propagation where the vector entry of $G(f)$ writes:
$$
 G_{m}(f)=e^{-2j\pi f r_{m}^{T}\theta}
$$
where $r_{m}$ is the 3D location of the sensor $m$ and $\theta$ the wave number of the planar wave. It follows that the coherence matrix of the $M$-ary observation has the following entry expression
\begin{eqnarray}
 \label{eq:delaymatrixpuredelay}
 C_{y,m\ell}(f)&=&e^{-2j\pi f (r_{m}-r_{\ell})^{T}\theta}
 \\ \nonumber
 &=&\int_{\mathds{R}^{3}} e^{-2j\pi f (r_{m}-r_{\ell})^{T}u}\delta_{\theta}(u)du
\end{eqnarray}
where $du$ is the infinitesimal Lebesgue measure in $\mathds{R}^{3}$ and $\delta_{\theta}$ the Dirac's distribution located in $\theta$. The expression \eqref{eq:delaymatrixpuredelay} can be seen as an expectation of $e^{-2j\pi f (r_{m}-r_{\ell})^{T}\theta}$ assuming that $\theta$ is deterministic.

%===================================================================
\subsubsection{Coherence model}

In a pioneer work Mack and Flinn \cite{mack_flinn:1971} propose to model the loss of coherence by considering that the azimuth, the elevation and velocity are uncertain. More specifically, authors in \cite{nouvellet:2012}  assume that the 3D vector $\mu=(a,e,c)$ where $a$ denotes the azimuth, $e$ the elevation and $c$ the sound velocity writes:
\begin{equation}
 \label{eq:randomnesswn}
 \mu=\mu_{0}+\nu
\end{equation} 
where $\mu_{0}=(a_{0},e_{0},c_{0})$ is a deterministic value and $\nu$ a zero-mean Gaussian random vector of dimension 3, whose covariance is denoted $\Sigma_{\mu}$. We know that $\mu$ is related to the 3D wavenumber vector $\theta$ by the one-to-one mapping 
\begin{eqnarray*}
 \label{eq:tehtaxyz}
 {\footnotesize
 \begin{array}{rcl}
 f:\,
  \mu= 
 \left\{
 \begin{array}{ll}
 a=\arg(\theta_{2}+j\theta_{1})&a\in(0,2\pi)
 \\
 e=\arg\sin(c\theta_{3})&e\in(-\pi/2,\pi/2)
 \\
 c=(\theta_{1}^{2}+\theta_{2}^{2}+\theta_{3}^{2})^{1/2}&c\in\mathds{R}^{+}
 \end{array}\right.
 &
 \longleftrightarrow\vspace{6pt}
 \\ 
 &&\hspace{-5.5cm}
 \,\theta=\left\{
 \begin{array}{ll}
 \theta_{1}=-c^{-1}\sin(a)\cos(e)
 \\
 \theta_{2}=c^{-1}\cos(a)\cos(e)&\in\mathds{R}^{3}
 \\
 \theta_{3}=c^{-1}\sin(e) 
 \end{array}\right.
 \end{array}}
\end{eqnarray*}
Using a first order Taylor's expansion, it follows that $\theta\approx f(\mu_{0})+J(\mu_{0})(\mu-\mu_{0})$ where the Jacobian writes
\begin{eqnarray*}
 &&J(\mu_{0}) =
 \\
 &&\hspace{-0.5cm}
 {\footnotesize
 \begin{bmatrix}
 -c_{0}^{-1}\cos(a_{0})\cos(e_{0})&
 c_{0}^{-1}\sin(a_{0})\sin(e_{0})&
 c_{0}^{-2}\sin(a_{0})\cos(e_{0})
 \\
 -c_{0}^{-1}\sin(a_{0})\cos(e_{0})&
 -c_{0}^{-1}\cos(a_{0})\sin(e_{0})&
 -c_{0}^{-2}\cos(a_{0})\cos(e_{0})
 \\
 0& 
 c_{0}^{-1}\cos(e_{0})&
 -c_{0}^{-2}\sin(e_{0})
 \end{bmatrix}}
\end{eqnarray*}
Therefore from \eqref{eq:randomnesswn} we can write $\theta\approx \theta_{0}+\varepsilon$ where $\varepsilon=J(\mu_{0})\nu$ appears as a Gaussian, zero-mean random vector of dimension 3 whose the covariance is
\begin{eqnarray}
 \label{eq:aec2theta}
\Sigma_{\theta} \approx J(\mu_{0}) \Sigma_{\mu} J'(\mu_{0})
\end{eqnarray}
Based on the expression \eqref{eq:delaymatrixpuredelay}, the coherence matrix entry of the model with random wavefront writes:
%\begin{eqnarray}
% \label{eq:Cfr1r2Phi}%\nonumber
%  C_{m,\ell}(f)&=&
%  \int_{\mathds{R}^{3}}  e^{-2j\pi f(r_{m}-r_{\ell})^{T}(\theta_{0}+e)} p_{\varepsilon}(e)de
%\end{eqnarray}
%We let It follows that 
\begin{eqnarray*}
   C_{m,\ell}(f)&=&e^{-2j\pi f(r_{m}-r_{\ell})^{T}\theta_{0}}
   \Phi_{\epsilon}(2\pi f(r_{m}-r_{\ell}))
\end{eqnarray*}
where $\Phi_{\epsilon}:u\in\mathds{R}^{3}\mapsto\esp{e^{ju^{H}\varepsilon}}\in\mathds{C}$ is the characteristic function of $\varepsilon$. 
Since $\varepsilon$ is Gaussian distributed with zero-mean and covariance matrix $\Sigma_{\theta}$, hence we have
\begin{eqnarray}
\label{eq:Cfr1r2Gaussian}
C_{m,\ell}(f)=
\underbrace{e^{-2j\pi f(r_{m}-r_{\ell})^{T}\theta_{0}}}_{\text{pure delay}}
 \underbrace{e^{-2\pi^{2}f^{2}(r_{m}-r_{\ell})^{T}\Sigma_{\theta}(r_{m}-r_{\ell})}}_{\text{LOC}}
 \end{eqnarray}
The first term corresponds to the pure delay which is associated to the deterministic part and whose the modulus is 1. The second term corresponds to the LOC associated to the stochastic part. Finally the log-MSC writes:
\begin{eqnarray}
 \label{eq:logcoherencemodel}
 \log\rho_{m,\ell}(f)=-2\pi^{2}f^{2}(r_{m}-r_{\ell})^{T}\Sigma_{\theta}(r_{m}-r_{\ell})
\end{eqnarray}
Let us notice that the log-MSC depends on the relative direction of the vector $(r_{m}-r_{\ell})$ w.r.t. the eigenvectors of $\Sigma_{\theta}$. 


We know that $\mu$ is related to the slowness vector $k$ by the one-to-one mapping 
\begin{eqnarray*}
 \label{eq:tehtaxyz}
 {\footnotesize
 \begin{array}{rcl}
 f:\,
  \mu= 
 \left\{
 \begin{array}{ll}
 a=\arg(k_{2}+jk_{1})&a\in(0,2\pi)
 \\
 e=\arg\sin(ck_{3})&e\in(-\pi/2,\pi/2)
 \\
 v=(k_{1}^{2}+k_{2}^{2}+k_{3}^{2})^{1/2}&v\in\mathds{R}^{+}
 \end{array}\right.
 &
 \Longleftrightarrow\vspace{6pt}
 \\ 
 &&\hspace{-5.5cm}
 \,k=\left\{
 \begin{array}{ll}
 k_{1}=-v^{-1}\sin(a)\cos(e)
 \\
 k_{2}=v^{-1}\cos(a)\cos(e)&\in\mathds{R}^{3}
 \\
 k_{3}=v^{-1}\sin(e) 
 \end{array}\right.
 \end{array}}
\end{eqnarray*}
Using a first order Taylor's expansion, it follows that $k \approx f(\mu_{0})+J(\mu_{0})(\mu-\mu_{0})$ where the Jacobian writes
\begin{eqnarray*}
 J(\mu_{0}) &=& 
 v_{0}^{-1}
 \begin{bmatrix}
 -\cos(a_{0})\cos(e_{0})&
 \sin(a_{0})\sin(e_{0})&
 v_{0}^{-1}\sin(a_{0})\cos(e_{0})
 \\
 -\sin(a_{0})\cos(e_{0})&
 -\cos(a_{0})\sin(e_{0})&
 -v_{0}^{-1}\cos(a_{0})\cos(e_{0})
 \\
 0& 
 \cos(e_{0})&
 -v_{0}^{-1}\sin(e_{0})
 \end{bmatrix}
\\
&=&
v_{0}^{-1} \times \tilde J(\mu_0)
\end{eqnarray*}
Therefore from \eqref{eq:randomnesswn} we derive $k\approx k_{0}+\varepsilon$ where $\varepsilon=J(\mu_{0})\,\nu$ appears as a zero-mean Gaussian random vector whose the covariance is:
\begin{eqnarray}
 \label{eq:aec2theta}
\Sigma_{k} \approx v_{0}^{-2} 
\underbrace{\tilde J(\mu_{0}) \Sigma_{\mu} \tilde J^{T}(\mu_{0})}_{\tilde \Sigma_{k}}
\end{eqnarray}
Hence the coherence matrix entry of the model with random wavefront writes:
\begin{eqnarray*}
   C_{m,\ell}(f)&=&e^{-2j\pi f(r_{m}-r_{\ell})^{T}k_{0}}
   \Phi_{\epsilon}(2\pi f(r_{m}-r_{\ell}))
\end{eqnarray*}
where $\Phi_{\epsilon}:u\in\mathds{R}^{3}\mapsto\esp{e^{ju^{T}\varepsilon}}\in\mathds{C}$ is the characteristic function of $\varepsilon$. 
Since $\varepsilon$ is Gaussian distributed with zero-mean and covariance matrix $\Sigma_{\theta}$, hence we have
\begin{eqnarray}
\label{eq:Cfr1r2Gaussian}
C_{m,\ell}(f)=
\underbrace{e^{-2j\pi f(r_{m}-r_{\ell})^{T}k_{0}}}_{\text{pure delay}}
 \underbrace{e^{-2\pi^{2}(f/v_0)^{2}(r_{m}-r_{\ell})^{T}
                 \tilde \Sigma_{k}(r_{m}-r_{\ell})}}_{\text{LOC}}
 \end{eqnarray}
