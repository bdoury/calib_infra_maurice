% !TEX root = ../distCjHMSC.tex
The full process consists of
\begin{itemize}
\item
 move data from the IDC, using {\tt execGPARSE.m}. The data with Matlab format is saved. This data consists of a structure called {\tt records} with the following items:
 \begin{itemize}
 \item
{\tt data}: sequence of samples,
 \item
{\tt Fs\_Hz}: sampling frequency,
\item
{\tt stime}, {\tt time}: start and end times, 
\item
{\tt station}: I26C1--I26C8 or I26H1--I26H8
\item
{\tt channel}: 'BDF' or 'LKO' or 'LWD' or 'LWS'
 \end{itemize}   
Typically in the following we use BDF on H and C.   The three other channels are only provided on H1. Channel LWD yields  the wind speed measurement. The wind sensor is located at about 2 meters above the infrasonic sensor. The frequency sampling depends on the channel. Typically for infrasound signals, the frequency sampling is $20$ Hz, and for wind features is 1 Hz.
    
\item
 provide the script {\tt filtercharacteristics.m}, following the example given in Annex.
 \item
 run the program {\tt ..... }.
\end{itemize}