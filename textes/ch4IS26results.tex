% !TEX root = ../calibreport.tex
%================================
The results of this section reports the estimation for the eight SUT given in the table \ref{tab:sensor-specifications}.
The analysis is on the data from IS26. Each query on the DB is on 2 consecutive days from the 1st of June to the 9th of October, saying about 130 days. For each file a spectral analysis is processed leading to an SUT response estimate. From the spectral components we perform, in each frequency band and for each time interval, the response of the system under test (SUT) using the known response of the reference sensor. These responses includes both the sensor and the digitizer. These responses are averaged over the full set of time intervals over the all days. The  MSC threshold is fixed to $0.98$. 

Let us remain the PTS specifications:
\begin{itemize}
\item
IMS pass-band requirement $[0.02 - 4]$ Hz;
 \item
$\pm 5\%$ on the response magnitude, i.e. $\pm 0.43$ in dB scale;
\item
the calibration is required at least once a year;
 \item
no requirement  on the phase but $\ldots$ $\pm 5\degree$ as for seismic requirements.
\end{itemize}


 \newpage
%================================
\section{Results of IS26 - sensor \#1}
%================================
The average estimate of the SUT response, interm of gain in dB and phase in degree, are reported figure \ref{fig:3monthsonIS26SUTboxplot1}.
%================================
\subsection{SUT response in the full band averaged on the full period of time}
%================================
\figscale{slidesITW2015/3monthsonIS26SUTboxplot1.pdf}{The red dashed lines report the theoretical SUT response range requirement, which is $\pm 0.43$ for the gain and $\pm 5\degree$ for the phase.}{fig:3monthsonIS26SUTboxplot1}{0.8}

%================================
\subsection{Full band on two successive days}
%================================
\figscale{slidesITW2015/averagedon20days1.pdf}{Sup ratio given by the expression \eqref{eq:ratio-sup-bis-exact}. The grey dots report the ratios averaged on a pair of 2 consecutive days, the red  disks report the average on 10 pairs of days. }{}{0.8}

%================================
\subsection{Full band for $20$ randomly selected days}
%================================
\figscale{slidesITW2015/averagedon20days1.pdf}{Sup ratio given by the expression \eqref{eq:ratio-sup-bis-exact}. The grey dots report the ratios averaged on a pair of 2 consecutive days, the red  disks report the average on 10 pairs of days. }{}{0.8}

%================================
\newpage\clearpage
%================================
\subsection{Stability at $1$ Hz}
%================================

%================================
\newpage\clearpage
%================================
%================================
\section{Comments}
%================================
\subsection{About the selected MSC threshold}
%================================
We said that we can not choose a too low value for the MSC threshold because the underdetermination. That appears clearly on figure \ref{fig:afewdaysonI26C6H6twoMSC}.
When the  MSC threshold is $0.7$, there is a large discrepancy between the estimated ratios, see \eqref{eq:estimated-Ratio} and we outline that there is no way to solve the underdetermination. But for an MSC threshold of $0.95$, the two curves are very close.

\figscale{afewdaysonI26C6H6twoMSC.pdf}{Couple H6C6, for two MSC thresholds.}{fig:afewdaysonI26C6H6twoMSC}{0.7}

On the other hand if the ratio between the two noise levels are perfectly known the indetermination is removed and we can use formula \eqref{eq:known-noise-ratio}. You might be tempted to say that the two noises on the two sensors are identical except theirs levels and consider that the ratio is given by the number of inlets in the noise reduction system. However the figure \ref{fig:afewdaysonI26C4H4knownnoiseratio} shows that the curve with a MSC threshold of $0.5$ and the formula \eqref{eq:known-noise-ratio} lead to values different from the curves obtained with a a MSC threshold of $0.95$.

\figscale{afewdaysonI26C4H4knownnoiseratio.pdf}{Couple H4C4, for two MSC thresholds with formula \eqref{eq:known-noise-ratio}.}{fig:afewdaysonI26C4H4knownnoiseratio}{0.7}

%================================
\newpage\clearpage
%================================
\subsection{Dip on the curves}
%================================
For the figures \ref{fig:C2} to \ref{fig:C5} associated to the MB2005, an important dip around 0.1 Hz is observed. 

Also we have reported figure \ref{fig:afewdays1colocation} a few days on the location H2C2. The different colors are for different days. The distribution seems to be uniform and does not depend on the day. The ratio seems to be different of 1.\\*

If the MSC is about 0.99, meaning that the noises on the two sensors are negligible, and if the two sensors have the same response, the only reason to get a ratio less than 1, is that the acoustical SOI on the SUT is attenuated, may due to the noise system reduction. That would write: it exists $\alpha\in\mathbb{C}$ with $|\alpha|<1$ s.t.:
\begin{eqnarray}
\label{eq:model-of-obervation}
\left\{
\renewcommand\arraystretch{1.6}
\begin{array}{rcl}
x_{\ut}(t)&=&g_{\ut}  \star (\alpha s(t))
\\
x_{\rf}(t)&=&g_{\rf}  \star s(t)
\end{array}
\right.
\end{eqnarray}

\figscale{afewdays1colocation.pdf}{a few days on H2C2. Only the band $[0.08-0.12]$ Hz is selected. The coherence is above $0.99$. The different colors for the different days.}{fig:afewdays1colocation}{0.5}


Another way to see the response differences between the 2 sensors in the band  $[0.08-0.12]$ Hz (gain ratio different of 1), appears figure \ref{fig:filteredsignals}. A zoom of the signals is plotted. It consists of about 1 minute, around a position where the observed coherence is above $0.99$. We see that the signal on the SREF  is bigger than this on the SUT. There is no way and no reason to reject this time window since the difference can be due either to the loss of gain or a unknown transfer function. 
\begin{figure}%{20cm}
\begin{minipage}{10cm}
              \includegraphics[scale=0.5]{signalsanomaly.pdf}
\end{minipage}
\begin{minipage}[c]{8cm}
              \includegraphics[scale=0.5]{filteranomaly.pdf}    

\end{minipage}
\centering
\caption{Filtered signals}
\label{fig:filteredsignals}
\end{figure}

