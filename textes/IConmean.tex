
% !TEX root = ../calibreport.tex
%==============================================================
\section{Trimmed mean}
%==============================================================
In the proposed procedure we compute an estimation of the Sup ratio by weighted averaging on a few number of spectral component period. On our process that is about $48$ hours for a maximum size of the spectral component period of $1000$ seconds, i.e. about $170$ samples. 

Taking several of such periods, we have to compute a mean. To get a robust averaging it is common to use the trimmed mean.
The classical estimator of the mean writes:
\begin{eqnarray*}
\hat \mu^{(1)}_{N} &=&\frac{1}{N}\sum_{n=1}^{N}X_{n}
\end{eqnarray*}
It is well known that this estimator is sensitive to the presence of outliers. In this case it is well known that the median is more robust. The median writes
\begin{eqnarray*}
\hat \mu^{(2)}_{N} &=&X_{(N/2)}
\end{eqnarray*}
where $X_{(n)}$ denotes the $n$-th value of the ordered sequence.

The trimmed mean is a compromise between the mean and the median. It consists to remove a given percent of the extreme values:
\begin{eqnarray*}
\hat \mu^{(3)}_{N} &=&\frac{1}{\lfloor N(1-2\alpha)\rfloor}\sum_{n=\lfloor \alpha N \rfloor }^{\lfloor N((1-\alpha)\rfloor }X_{n}
\end{eqnarray*}
where $\alpha$ is between $0$ and $0.5$

%==============================================================
\section{Confidence interval on the mean}
%==============================================================

We consider a sequence of $N$ data modeled as $N$ i.i.d. r.v. denoted $X_{n}$. An estimator of the mean is given by
\begin{eqnarray*}
\hat \mu_{N} &=&\frac{1}{N}\sum_{n=1}^{N}X_{n}
\end{eqnarray*}
We let
\begin{eqnarray*}
\hat\sigma_{N}^{2} &=& \frac{1}{N-1}\sum_{n=1}^{N}(X_{n}-\hat \mu)^{2}
\end{eqnarray*}
To provide a confidence interval (CI) for $\hat\mu_{N}$ we have
\begin{itemize}
\item
if $X_{n}$ are gaussian with mean $\mu$ and variance $\sigma^{2}$ both unknown, it is show that the CI can be derived from the Student distribution with $N-1$ d.o.f. more specifically we have:
\begin{eqnarray*}
\frac{\sqrt{N}(\hat \mu_{N} -\mu)}{\hat\sigma_{N}}&\sim&T_{N-1}
\end{eqnarray*}
leading to the CI:
\begin{eqnarray*}
\hat\mu_{N}-\alpha\frac{\hat\sigma_{N}}{\sqrt{N}}
&
 \leq\mu\leq
&
\hat\mu_{N}+\alpha\frac{\hat\sigma_{N}}{\sqrt{N}}
\end{eqnarray*}

\item
for large $N$ (limit central theorem), it is shown that 
\begin{eqnarray*}
\frac{\hat\sigma_{N}}{\sqrt{N}}
\end{eqnarray*}
leading to the CI:
\begin{eqnarray*}
\hat\mu_{N}-\beta\frac{\hat\sigma_{N}}{\sqrt{N}}
&
 \leq\mu\leq
&
\hat\mu_{N}+\beta\frac{\hat\sigma_{N}}{\sqrt{N}}
\end{eqnarray*}

\item
for limit central theorem, 
\end{itemize}








